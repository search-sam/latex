\documentclass[letterpaper]{report}

\usepackage[spanish]{babel}
\usepackage[T1]{fontenc}
\usepackage{pifont}

%% Portada del Libro
\title{Catequesis del Papa Francisco sobre el\\ <<\textit{Noviasgo - Matrimonio - Familia}>>}
\author{Recopilacion Catequetica} 
\date{ \textit{Transcripciones de\\ Audiencias Generales\\ 10 Diciembre 2014 a 19 Agosto 2015} }
	
\begin{document}
	%% - Reglas especificas para este libro
	%\renewcommand{\labelitemi}{\ding{108}}
	%\renewcommand{\thesection}{\thechapter}
	%\renewcommand{\chaptername}{algo}
	\setcounter{secnumdepth}{0}
	
	%% - Portada
	\maketitle
	
	%% - Indice
	\tableofcontents
	
	%% - Introduccion (Seccion 1)
	\section{ Introducci\'on }
		Queridos hermanos y hermanas, !`Buenos d\'ias!.\\
		
		Hemos concluido un ciclo de catequesis sobre la Iglesia. Agradecemos al Se\~nor 
		que nos ha hecho recorrer este camino descubriendo la belleza y la responsabilidad de
		pertenecer a la Iglesia, de ser Iglesia todos nosotros.
		
		Ahora iniciamos una nueva etapa, un nuevo ciclo y el tema ser\'a la familia, que se
		integra en este tiempo intermedio entre dos Asambleas del Sínodo dedicadas a esta
		realidad tan importante. Por esto, antes de entrar en el recorrido sobre los diversos
		aspectos de la vida familiar, hoy deseo partir precisamente de la Asamblea Sinodal
		del pasado mes de octubre, que tenía este tema: “Los desafíos pastorales sobre la
		familia en el contexto de la nueva evangelización”. Es importante recordar cómo se
		desarrolló y qué produjo. Cómo fue y qué produjo.

		Durante el Sínodo los Medios han realizado su trabajo – había mucha expectativa,
		mucha atención – y les agradecemos porque lo hicieron también con abundancia.
		¡Tantas noticias, tantas! Esto fue posible gracias a la Oficina de Prensa, que cada
		día hizo un briefing. Pero a menudo la visión de los Medios era un poco en el estilo
		de las crónicas deportivas o políticas: se hablaba frecuentemente de dos equipos, 
		pro y contra, conservadores y progresistas, etc. Hoy quisiera relatar lo que ha sido
		el Sínodo.
		
		En primer lugar, yo les pedí a los Padres sinodales que hablaran con franqueza y
		coraje y que escucharan con humildad, que dijeran todo lo que tenían en el corazón
		¡con coraje! En el Sínodo no hubo censura previa, no hubo. Cada uno podía, es más,
		debía decir lo que tenía en el corazón, lo que pensaba sinceramente. “Pero padre,
		sesto creará discusión”.
		
		Es verdad, hemos escuchado cómo discutieron los apóstoles, el texto dice: “surgió una
		fuerte discusión”. Gritaban entre ellos los apóstoles, ¡sí! Porque buscaban la
		voluntad de Dios sobre los paganos, si podían entrar en la Iglesia o no. 
		Era una cosa nueva.

		Siempre cuando se busca la voluntad de Dios en una asamblea sinodal hay diversos
		puntos de vista y discusión. ¡Y aquello no es una cosa fea! Siempre que se haga con
		humildad y con ánimo de servicio a la asamblea de los hermanos. Hubiera sido una cosa
		mala la censura previa. No, no. Cada uno debía decir lo que pensaba.
		
		Después de la Relación inicial del Card. Erdö, hubo un primer momento, fundamental,
		en el cual todos los Padres pudieron hablar y todos escucharon. Y era edificante
		aquella actitud de escucha que tenían los Padres.
		
Un momento de gran libertad, en el cual cada uno expuso su pensamiento con parresía y con confianza. Como base de las intervenciones estaba el “Instrumento de trabajo”, fruto de la precedente consulta de toda la Iglesia. Y aquí tenemos que agradecer a la Secretaría del Sínodo por el gran trabajo que realizó ya sea antes que durante la Asamblea. De verdad han sido muy buenos.
Ninguna intervención puso en discusión las verdades fundamentales del Sacramento del Matrimonio, ninguna intervención, es decir: la indisolubilidad, la unidad, la fidelidad y la apertura a la vida. Esto no ha sido tocado.
Todas las intervenciones fueron recogidas y así se llegó al segundo momento, es decir, un borrador que se llama la Relación después de la discusión. También esta Relación fue desarrollada por el Cardenal Erdö, articulada en tres puntos: la escucha del contexto y de los desafíos de la familia; la mirada fija en Cristo y el Evangelio de la familia; la confrontación con las perspectivas pastorales.
Sobre esta primera propuesta de síntesis se llevó a cabo la discusión en los grupos, que fue el tercer momento. Los grupos, como siempre, estaban separados por idiomas, porque es mejor así, se comunica mejor: italiano, inglés, español y francés. Cada grupo al final de su trabajo presentó una relación y todas las relaciones de los grupos fueron publicadas inmediatamente. Todo fue dado, había transparencia para que se supiera lo que sucedía.
A ese punto –el cuarto momento– una comisión examinó todas las sugerencias que surgieron de los grupos lingüísticos y se realizó la Relación final, que mantuvo el esquema anterior - escucha de la realidad, la mirada en el Evangelio y el compromiso pastoral - pero ha tratado de acoger el fruto de las discusiones en los grupos. Como siempre, también fue aprobado un Mensaje final del Sínodo, más breve y más divulgativo respecto a la Relación.
Este fue el desarrollo de la Asamblea sinodal. Algunos de ustedes pueden preguntarme: “Pero, padre, ¿han peleado los padres?” No sé si han peleado, pero que han hablado fuerte sí, de verdad. Y esta es la libertad, es justamente la libertad que hay en la Iglesia. Todo ocurrió “cum Petro et sub Petro”, es decir, con la presencia del Papa, que es garantía para todos de libertad y de confianza, y garantía de la ortodoxia. Y al final, con una intervención mía, di una lectura sintética de la experiencia sinodal.
Por lo tanto, los documentos oficiales que salieron del Sínodo son tres: el Mensaje final, la Relación final y el Discurso conclusivo del Papa. No hay otros.
La Relación final, que fue el punto de llegada de toda la reflexión de las diócesis hasta aquel momento, fue publicada ayer y se envía a las Conferencias Episcopales, que la discutirán en vistas de la próxima Asamblea, aquella Ordinaria, en octubre de 2015.
Digo que ayer fue publicada – ha sido publicada antes- pero ayer fue publicada con las preguntas que se hacen a las conferencias episcopales: se convierte en Lineamenta del próximo Sínodo.
Debemos saber que el Sínodo no es un parlamento: viene el representante de esta Iglesia, de esta Iglesia, de aquella Iglesia…No, no es esto. Viene sí, un representante, pero la estructura no es parlamentaria, es totalmente diferente: el Sínodo es un espacio protegido para que el Espíritu Santo pueda obrar; no hubo enfrentamiento entre facciones, como en el parlamento –eso es lícito en un parlamento–  sino un confronto entre los Obispos, que llegó después de un largo trabajo de preparación y que ahora continuará en otro trabajo, para el bien de la familia, de la Iglesia y de la sociedad. Es un proceso, es el normal camino sinodal.
Ahora esta Relatio, regresa a las Iglesias particulares y continúa en esas Iglesias el trabajo de oración, reflexión y discusión fraterna con el fin de preparar la próxima Asamblea. Esto es el Sínodo de los Obispos. Lo confiamos a la protección de la Virgen, nuestra Madre.
Que ella nos ayude a seguir la voluntad de Dios tomando las decisiones pastorales que ayuden más y mejor a la familia. Les pido que acompañen este proceso sinodal, hasta el próximo Sínodo, con la oración. Que el Señor nos ilumine, nos haga ir hacia la madurez de aquello que debemos decir a todas las Iglesias como Sínodo. Y para esto es importante su oración. Gracias.
	
	%% - Conexion entre el amor conyugal y la transmicion de la vida (Seccion 2)
	{\centering 
		\section
		[CONEXI\'ON ENTRE EL AMOR CONYUGAL Y LA TRANSMISI\'ON DE LA VIDA]
		{
			{\normalsize CONEXI\'ON ENTRE EL AMOR CONYUGAL Y LA TRANSMISI\'ON DE LA VIDA
			\footnote{
				Catequesis del Papa en la \textit{Audiencia general} 
				del mi\'ercoles,11 de julio de 1984.
			}}
		}
	}
	 
		\subsection{{\normalsize Un importante acto del Magisterio de la Iglesia: 
		 La Enc\'iclica <<\textit{Humanae vitae}>>}}
			1. Las reflexiones que hasta ahora hemos expuesto 
			acerca del amor humano en el plano divino, quedar\'ian, 
			de alg\'un modo, incompletas si no trat\'asemos de ver su 
			aplicaci\'on concreta en el \'ambito de la moral conyugal y 
			familiar. Deseamos dar este nuevo paso, que nos llevar\'a 
			a concluir nuestro ya largo carnino, bajo la gu\'ia de una 
			importante declaraci\'on del Magisterio reciente: la Enc\'iclica 
			\textit{\textit{Humanae vitae}}, que public\'o el Papa Pablo VI, en julio 
			de 1968. Vamos a releer este significativo documento a la 
			luz de los resultados a que hemos llegado, examinando el 
			designio inicial de Dios y las palabras de Crista, que nos 
			remiten a \'el.
			
		\subsection{Todo acto matrimonial debe permanecer abierto a la transmisi\'on
		 de la vida}
			2. <<La Iglesia... \textit{ense\~na} que cualquier acto matrimonial 
			debe quedar abierto a la transmisi\'on de la vida... >> 
			(\textit{\textit{Humanae vitae}}, 11). <<Esta doctrina, muchas veces expuesta 
			por el Magisterio, esta fundada sobre la inseparable conexi\'on 
			que Dios ha querido y que el hombre no puede 
			romper por propia iniciativa, entre \textit{los dos significados del 
			acto conyugal}: el significado unitivo y el significado 
			procreador>> (\textit{\textit{Humanae vitae}}, 12).
			
		\subsection{Inseparabilidad del acto unitivo y procreador}
			3. Las consideraciones que voy a hacer se referir\'an 
			especialmente al pasaje de la Enc\'iclica \textit{Humanae vitae}, que 
			trata de los <<dos significados del acto conyugal>> y de su 
			<<inseparable conexi\'on>>. No intento hacer un comentario a 
			toda la Enc\'iclica, sino m\'as bien explicarla y profundizar en 
			dicho pasaje. Desde el punto de vista de la doctrina moral 
			contenida en el documento citado, este pasaje tiene un 
			significado central. Al mismo tiempo es un p\'arrafo que se 
			relaci\'ona estrechamente con nuestras anteriores reflexiones 
			sobre el \textit{matrimonio en su dimension de signa (sacramental)}.
			 
			Puesto que, seg\'un he dicho, se trata de un pasaje central 
			en la Enc\'iclica. resulta obvio que est\'e inserto muy 
			profundamente en .toda su estructura: su an\'alisis, en 
			consecuencia, debe orientarse hacia las diversas componentes 
			de esa estructura, aunque la intenci\'on no sea comentar 
			to do el texto.
			
			4. En las reflexiones acerca del signa sacramental, se 
			ha dicho ya varias veces que est\'a basado sobre \textit{<<el 
			lenguaje del cuerpo>> releido en la verdad}. Se trata de una verdad 
			afirmada por primera vez al principio del matrimonio, 
			cuando los nuevos esposos, prometiendose mutuamente 
			<<ser fieles siempre... y amarse y respetarse durante todos 
			los dias de su vida>>, se convierten en ministros del 
			matrimonio como sacramenta de la Iglesia.
			 
			Se trata, por tanto, de una verdad que por decirlo as\'i, 
			se afirma siempre de nuevo. En efecto, el hombre y la 
			mujer, viviendo en el matrimonio <<hasta la muerte>>, 
			reproponen siempre, en cierto sentido, ese signa que ellos 
			pusieron --a trav\'es de la liturgia del sacramento-- 
			el d\'ia de su matrimonio.
			 
			Las palabras antes citadas de la Enc\'iclica del Papa 
			Pablo VI se refieren a ese momento de la vida com\'un de los 
			c\'onyuges, en el cual, al unirse mediante el acto conyugal, 
			ambos vienen a ser, seg\'un la expresion b\'iblica, <<una sola 
			came>> (\textit{Gen} 2, 24), Precisamente en \textit{ese momento tan rico 
			de significado}, es tambi\'en particularmente importante 
			que se relea el <<lenguaje del cuerpo>> en la verdad. Esa 
			lectura se convierte en condici\'on indispensable para actuar 
			en la verdad, o sea, para comportarse \textit{en conformidad con 
			el valor y la norma moral}.
			
			5. La Enc\'iclica no s\'olo recuerda esta norma, sino que 
			intent a tambi\'en darle \textit{su fundamento adecuado}. Para 
			aclarar m\'as a fondo esa <<inseparable conexi\'on que Dios ha 
			querido... entre los dos significados del acto conyugal>>, 
			Pablo VI contin\'ua asi en la frase siguiente: <<...el acto 
			conyugal, por su \'intima estructura, mientras une profundamente 
			a los esposos. los hace aptos para la generaci\'on de 
			nuevas vidas, seg\'un las leyes inscritas en el ser mismo del 
			hombre y de la mujer>> (\textit{Humanae vitae}, 12).
			 
			Podemos observar c\'omo en la frase precedente el 
			texto reci\'en citado trata, sobre todo, del <<\textit{significado}>> 
			y en la frasesucesiva, de la <<\textit{\'intima estructura}>> 
			(es decir, de la naturaleza) de la relaci\'on conyugal. 
			Definiendo esta <<\'intima estructura>>, 
			el texto hace referencia alas <<leyes inscritas 
			en el ser mismo del hombre y de la mujer>>.
			
			El paso de la frase, que expresa la norma moral, a la 
			frase que la explica y motiva, es particularmente significativo. 
			La Enc\'iclica nos induce a buscar el fundamento 
			de la norma, que determina la moralidad de las acciones 
			del hombre y de la mujer en el acto conyugal, en la 
			naturaleza de este mismo acto y, todav\'ia mas profundamente, 
			en la naturaleza de \textit{los sujetos mismos} que act\'uan.
			
		\subsection{El acto conyugal une a los esposos y los hace aptos 
		 para engendrar nuevas vidas}
			6. De este modo, la «intima estructura» (0 sea, la na- 
			turaleza) del acto conyugal constituye la base necesaria 
			para una adecuada lectura y descubrimiento de los signi- 
			[icados, que deben ser transferidos a la conciencia y alas 
			decisiones de las personas agentes, y tambien la base ne- 
			cesaria para establecer la adecuada relacion entre estos 
			significados, es decir, su inseparabilidad. Dado que, «el 
			acto conyugal.,» -a un mismo tiempo- «une profunda- 
			mente a los esposos», y, a la vez, «105 hace aptos para la 
			genera cion de nuevas vidas»: y tanto una cosa como otra 
			se realizan «por su intima estructura»: de todo se deriva 
			en consecuencia que la persona humana (con la necesi- 
			dad pro pia de la razon, la necesidad logica) «de be» leer 
			contemporaneamente los «dos signiiicados del acto con- 
			yugal» y tarnbien la «inseparable conexion. .. entre los dos 
			significados del acto conyugal». 	< 
			No se trata, pues, aqui de ninguna otra cosa sino de 
			leer en la verdad el «lenguaje del cuerpo», como repetidas 
			veces hemos dicho en los precedentes analisis biblicos. La 
			norma moral, ensefiada constantemente por la Iglesia en 
			este ambito r Y recordada y reafirmada por Pablo VI en 
			su Enciclica, brota de la lectura del «lenguaje del cuerpo» 
			en la verdad. 
			Se trata aqui de la verdad, primero en su dimension 
			ontologica (<<estrutura intima>>) y Iuego -=-~n con.sec:~en- 
			cia- de la dimension subjetiva Y psicologica (esignifica- 
			do»). El texto de la Enciclica sub ray a que, en el caso en 
			cuestion, se trata de una norma de la ley natural.
	
	{\centering
		\section{LA NORMA MORAL SOBRE LA VIDA MATRIMONIAL}
	}
		1. En la Enciclica Humanae vitae leemos: «AI exigir 
		que los hombres observen las normas de la ley natural, in- 
		terpretada por su constante doctrina, la Iglesia ensefia que 
		cualquier acto matrimonial debe quedar abierto a la 
		transmision de la vida» (Humanae vitae, 11). 
		Conternporaneamente el mismo texto considera e in- 
		cluso pone de relieve la dimension subjetiva y psicologi- 
		ca, al hablar de «significado», y exactamente, de los «dos 
		significados del acto conyugal». 	< 
		El signiiicado surge en la conciencia con la relectura 
		de la verdad (ontologia) del objeto. Mediante esta relectu- 
		ra, la verdad (ontologica) entra, por asi decirlo, en la di- 
		mension cognoscitiva: subjetiva y psicologica. 
		La Humanae vitae parece dirigir particularmente 
		nuestra intencion ha cia est a ultima dimension. Esto se 
		confirm a por 10 demas, indirectarnente, tambien con la 
		frase siguiente: «Nos pensamos que los hombres, en par- 
		ticular los de nuestro tiernpo, se encuentran en grado de 
		comprender el caracter profundamente razonable y hu- 
		< mano de este principio fundamental» (Humanae vitae, 12). 
		2. Este «caracter razonable» hace referencia no solo a 
		la verdad en la dimension ontologica, 0 sea, a 10 que corres- 
		ponde a la estructura real del acto conyugal. Se refiere
		tambien a-la misma verdad en su dimension subjetiva y 
		psicologica, es decir, a la recta comprension de la intima es- 
		tructura del acto conyugal, 0 sea, a la adecuada relectura 
		y de su inseparable conexion, e~ orden .a una conducta 
		moralmente recta. En esto consiste precisamente la nor- 
		ma moral y la correspondiente ~egulacion de los ,:ctos h~- 
		manos en la esfera de la sexuahdad. En este sentido, deci- 
		mos, que la norma moral se identifica con la relectura, en 
		la verdad, del «lenguaje del cuerpo».  
		2 Audiencia general, 18- VII-1984.
		
		\subsection{La norma moral es de ley natural.\\ Esta en conformidad con la razon}
			3. La Enciclica Humanae vitae contiene por tanto, l~ 
			norma moral y su motivacion 0, ':11 m~-"os, una profundi- 
			zacion de 10 que constituye la motivacion de la norma. Por 
			otra parte, dado que en la nor~a se expresa de man era 
			vinculante el valor moral. se sigue de ello que los actos 
			conformes a la norma son moralmente rectos; .y.e~ cam- 
			bio, los actos contrarios, son intrinsecamente ilicitos, El 
			autor de la Enciclica subraya que tal norma pertenece a 
			la «ley natural», es decir~ que e~ta en conforrnidad con la 
			razon como tal. La Iglesia ensena esta norma, aunque no 
			este expresada formalmente (es decir, lit~ra~J?ente) en-la 
			Sagrada Escritura; y 10 hace con la conviccion de que la 
			interpretacion de los precept.os d~ la ley natural pertene- 
			ce a la competencia del Maglste.no. . 
			Podemos, sin embargo, decir mas. Aunque ~a norma 
			moral. formulada asi en la Enciclica H,,!mana~ VItae, no se 
			halla literalmente en la Sagrada Escritura, SIn embargo, 
			por el hecho de estar contenida en ~a Tradicion y -como 
			escribe el Papa Pablo VI- haber sido «otras. muchas ve- 
			ces expuesta por el Magisterio» (Humanae vitae, 1.2) a los 
			fieles, resulta que esta norma corresponde al cc:nJ.unto de 
			la doctrina revelada contenida en las fuentes biblicas (cfr 
			Humanae vitae, 4).
		
		\subsection{Esta en consonancia con el conjunto de la doctrina moral revelada}
			4. Se tratra aqui no solo del conjunto de la doctrina 
			moral contenida en la Sagrada Escritura, de sus premisas 
			esenciales y del caracter general de su contenido, sino 
			tambien.de ese conjunto mas amplio, al que hemos dedi- 
			cado anteriormente numerosos analisis, al tratar de la 
			«teologia del cuerpo». 
			Propiarnente, desde el fondo de este amplio conjunto, 
			resulta evidente que la citada norma moral pertenece no 
			solo a la ley moral natural. sino tambien al orden moral 
			revelado por Dios: tambien desde este punto de vista ello 
			no podriaser de otro modo, sino unicarnente tal cuallo 
			han transmitido la tradicion y el magisterio y, en nuestros 
			dias, la Enciclica Humanae vitae, como documento con- 
			ternporaneo de este magisterio. 
			Pablo VI escribe: «No pensamos que los hombres, en 
			particular los de nuestro tiernpo, se encuentran en grado 
			de comprender el caracter profundamente razonable y 
			humano de este principio fundamental» (Humanae vitae, 
			12). Podemos afiadir: ellos pueden comprender, tarnbien, 
			su profunda conformidad con todo 10 que transmite la 
			Tradicion, derivada de las fuentes biblicas. Las bases de 
			esta conformidad deben buscarse particularmente en la 
			antropologia biblica. Por otra parte, es sabido el significa- 
			do que la antropologia tiene para la etica, 0 sea, para la 
			doctrina moral. Parece, pues, que es del todo razonable 
			buscar precisamente en la «teologia del cuerpo» el [unda- 
			mento de la verdad de las normas que se refieren a la pro- 
			blernatica tan fundamental del hombre en cuanto «cuer- 
			PO»: «los dos seran una misma came» (Gen 2,24). 
			5. La norma de la Enciclica Humanae vitae afecta a 
			todos los hombres, en cuanto que es una norma de la ley 
			natural y se basa en la conformidad con la razon huma- 
			na (cuando esta, se entiende, busca la verdad). Con ma- 
			yor razon ella concieme a todos los fieles, miembros de 
			la Iglesia, puesto que el caracter razonable de esta norma 
			encuentra indirectamente confirrnacion y solido sos ten en 
			el conjunto de la «teologia del cuerpo». Desde este punto 
			de vista hemos hablado, en anteriores analisis, del «ethos» 
			de la redencion del cuerpo. 
			La norma de la ley natural, basada en este «ethos», en- 
			cuentra no solamente una nueva expresion, sino tambien 
			un fundamento mas pleno antropologico y etico, bien sea 
			en la palabra del Evangelio, bien sea en la accion purifi- 
			cante y fortificante del Espiritu Santo. 	. 
			Hay, pues, razones suficientes para que los creyentes 
			y, en particular, los teologos relean y comprendan cada 
			vez mas profundamente la doctrina moral de la Enciclica 
			en este contexto integral. 
			Las reflexiones, que des de hace tiempo venimos ha- 
			ciendo, constituyen precisamente un intento de una relec- 
			tura asi. 
	
	{\centering
		\section{ARMON\'IA ENTRE LA <<\textit{HUMANAE VITAE}>> Y LA <<GAUDIUM ET SPES>>}
	}
		\subsection{No hay contradlccion entre las leyes que regulan 
		la transmlslon de la vida y el verdadero amor humano }
			1. Reanudamos las reflexiones que tienden a armoni- 
			zar la Enciclica Humanae vitae con el conjunto de la teo- 
			logia del cuerpo. 
			Esta Enciclica no se limita a recordar la norma moral 
			que concierne a la eo nviv en cia conyugal, reafirmandola 
			ante las nuevas circunstancias. Pablo VI, al pronunciarse 
			con magisterio autentico mediante la Enciclica (1968), ha 
			tenido delante de sus ojos la autorizada enunciacion del 
			Concilio Vaticano IT, contenida en la Constituci6n Gau- 
			dium et spes (1965). 
			La Enciclica, no s610 se halla en la linea de la ense- 
			fianza conciliar, sino que constituye tambien el desarrollo 
			y la contemplacion de los problemas alli incluidos, de un 
			modo especial con referencia al problema de la «arrnonia 
			del amor humano con el respeto a la vida», Sobre este 
			punto, le em os en la Gaudium et spes las siguientes pala- 
			bras: «La Iglesia recuerda que no puede haber contradic- 
			cion verdadera entre las leyes divinas de la transmisi6n 
			3 Audiencia general, 25- VII-1984. 

			obligatoria de la vida y el fomento genuino del amor 'con- 
				yugal» (Gaudium et spes, 51). 	. 
			2. La Constituci6n Pastoral del Vaticano II excluye 
			toda «verdadera contradiccion», en el orden norrnativo, 10 
			cual, por su parte, confirma Pablo VI, procurando a la vez 
			proyectar luz sobre aquella «no-contradiccion» y, de ese 
			modo, motivar la respectiva norma moral, demostrando 
			la conformidad de la misma con la raz6n. 
			Sin embargo, la Humanae vitae habla no tanto de la 
			«no contradiccion» en el orden normativo, cuanto de la 
			«inseparable conexion» entre la transmisi6n de la vida y 
			el autentico am or conyugal desde el punto de vista de los 
			«dos significados del acto conyugal: el significado unitivo 
			y el significado procreativo» (Humanae vitae, 12), de los 
			cuales ya hemos tratado.
			
		\subsection{Es cierto que la doctrina de la regulaci6n de la natalidad 
		plantea a veces dificultades serias}
			3. Nos podriarnos detener largamente sobre el anali- 
			sis de la norma misma; pero el caracter de uno y otro do- 
			cumento lleva, sobre todo, a reflexiones, al menos indirec- 
			tamente, pastorales. En efecto, la Gaudium et spes es una 
			Constituci6n Pastoral, y la Enciclica de Pablo VI -con 
			todo su valor doctrinal- intenta tener la misma orienta- 
			ci6n. Quiere ser, efectivamente, respuesta a los interrogan- 
			, tes del hombre contemporaneo. Son, estos, interrogantes 
			de caracter demografico y, en consecuencia, de caracter 
			socio-econ6micoy politico, relacionados con el crecimien- 
			to de lapoblacion en el globo terrestre. Son interrogantes 
			que surgen en el campo de las ciencias particulares, y del 
			mismo estilo son los interrogantes de los moralist as con- 
			temporaneos (teologos-moralistas). Son antes que nada 
			los interrogantes de los c6nyuges, que se encuentra ya en 
			el centro de la atencion de la Constituci6n conciliar y que
			la Enciclica toma de nuevo con toda la prec~si6n que es 
			de desear. Precisamente leemos en ella: «Consideradas las 
			condiciones de la vida actual y dado el significadoque las 
			relaciones conyugales tienen en orden a la ~r~o~la entre 
			los esposos y a su mutua fidelidad, (no sena indicado re~ 
			visar las normas eticas hasta ahora vigentes, sobre t~do SI 
			se considera que las mismas no pueden observarse sm sa- 
			crificios, algunas veces heroicos?» (Humanae vitae, 3).
			
		\subsection{Se cuenta con la gracia de Dios y la buena voluntad de los esposos}
			4. En la antedicha formulaci6n es evidente la solici- 
			tud con la que el autor de la Enciclica p;-ocura afrontar 
			los interrogantes del hombre contemporaneo en todo su 
			alcance. El relieve de estos interrogantes supone una res: 
			puesta proporcionalmente ponder ad a y profunda. Pu.e~ ;'1, 
			por una parte, es justo esperar una y;ofunda exposicion 
			de la norma, por otra parte, nos es licito esperar que una 
			importacia no menor se conceda a los tem~ pastorales, ya 
			que conciernen mas directamente a la vida de los hom- 
			bres concretos, de aquellos, preci~a~~nte, que se plantean 
				las preguntas men cion ad as al pnnpplO. 	, 
			Pablo VI ha tenido siempre delante de .SI ~ estos horn- 
			bres. Expresi6n de ello es, entre. otros, ~l slgm.ente pasaje 
			de la Humanae vitae: «La doctnna de la Iglesia en mate- 
			ria de regulaci6n de la natalidad, pr?mulgadora de l~ !ey 
			divina, aparecera facilmente a .los ojos de much os dificil 
			e, incluso, imposible en la p:acllca. Y.en verdad que, co~o 
			todas las grarides y beneficiosas realidades, ~xz~e u!1 ;>eno 
			empefio y muchos esfuer~os de.orden familiar, individual 
			y social. Mas aun, no sena posible actuarla sm la ayuda 
			de Dios, que sostiene y fortalece la bu~na voluntad de los 
			hombres. Pero a todo aquel que reflexione seriamente, no 
			puede menos de aparecer que tales esfuerzos ennoblecen 
			al hombre y benefician la comunidad humana» (Humanae 
			vitae, 20).
			
		\subsection{Posibilidad de la observancia de la Iey divina}
			5. A est a altura no se habla mas de la no-contradic- 
			cion norrnativa, sino sobre todo de la «posibilidad de la ob- 
			servancia de la ley divino», es decir, de un tema al menos 
			indirectamente, pastoral. El hecho de que la ley'tenga que 
			ser de «posible» puesta en practica, pertenece directamen- 
			te ~ la misma naturaleza de la ley y esta, por tanto, con- 
			temdo en el cuadro de la «no-contradictoriedad normati- 
			va». Sin embargo, la «posibilidad», entendida coma actua- 
			lidad de la no.rma, pertenece tambien a la esfera practica 
			y pastoral. MI predecesor habla en el texto citado, preci- 
			samente, de este punto de vista. 
			. 6. Se puede afiadir aqui una consideracion: el hecho 
			de que toda la retrovision biblica; denominada «teologia 
			del cuerpo», nos ofrezca tambien, aunque indirect amen- 
			te, la confirmacion de la verdad de la norma moral con- 
			tenida en la Humanae vitae, nos predispone a considerar, 
			mas a [ondo, los aspectos priicticos y pastorales del proble- 
			ma en su conjunto. Los principios y presupuestos genera- 
			les de la «teologia del cuerpo», (no estaban, quiza sacados 
			todos ellos de las respuestas que Cristo dio alas pregun- 
			tas de sus concretos interlocutores? Y los textos de Pablo 
			-como, por ejemplo, los de la Carta a los Corintios-, (no 
			son, ac~so, un pequefio manual en orden a los problem as 
			de la vida moral de los primeros seguidores de Cristo? Y 
			en estos textos encontramos ciertamente, esa «norma de 
			cornprension» que parece tan indispensable frente a los 
			problemas que trata la Humanae vitae, y que est a presen- 
			te en esta Enciclica. 
			Si alguien cree que el Concilio y la Enciclica no tienen 
			bastante en cuenta las dificultades presentes en la vida 
			concreta, es porque no comprende las preocupaciones 
			pastorales que hubo en. el .D.rigen, de tales documentos. 
			Preocupacion pastoral significa busqueda ~el verdadero 
			bien del hombre, promocion de los valores impresos por 
			Dios en la propia persona; es decir; ~ignifica.Ia puest~ en 
			acto de «aquella regla de cornpr'eriston» que mten~a ~lem- 
			pre el decubrimiento cada vez mas claro del designio de 
			Dios sobre el amor humano, con la certeza de que el uni- 
			eo y verdadero bien de la persona humana consiste en la 
			realizacion de este designio divino. 
			Se podria decir que, precisa~::nte, en no.~bre de la 
			mencionada «norma de cornprension», el Concilio ha plan- 
			teado la cuestion de la «armonia del amor humano con el 
			respeto a la vida«) (Gaudium et spes, 51), y la Enciclica Hu- 
			manae vitae, no solo ha recordado luego las norm as mo- 
			rales que obligan a este.ambito, sino que s~ '?~upa ade- 
			mas, ampliamente, del problem a de la «posibilidad de la 
				observancia de la ley divina». 	. 
			Estas reflexiones actuales sobre el caracter del docu- 
			mento Humanae vitae nos preparan para tratar a conti- 
			nuacion el tema de la «paternidad responsable». 
	
	{\centering
		\section{LA PATERNIDAD Y LA MATERNIDAD RESPONSABLES}
	}
		\subsection{El caracter moral del comportamiento de los esposos 
		debe estar guiado por criterios objetivos}
			1. Hernos elegido para hoy el tema de la «paternidad 
			y maternidad responsables», a la Iuz de la Constitucion 
			Gaudium et ~pes y de la Enciclica Humanae vitae. . 
			La Constitucion conciliar, al afrontar el tema se limi- 
			ta a recordar las p~emisas ,fund~mentales; el do~umento 
			pontijicio, en cambio, va mas alla, dando a estas premisas 
			un os contenidos mas concretos. 
			.El texto conciliar dice asi: « ... Cuando se trata, pues, de 
			conjugar el amor conyugal con la responsable transmi- 
			sion de la vida, la Indole moral de: la conducta no depen- 
			de s?lame!lte de la smcera intencion y apreciacion de los 
			motivos, smo que debe determinarse con criterios objeti- 
			vos, to.ma~os de la naturaleza de la persona y de sus ac- 
			tos, cntenos que mantienen integro el sentido de la mu- 
			tua entrega y de la humana procreacion, entretejidos con 
			el amor verdadero: esto es imposible sin cultivar sincera- 
			mente la virtud de la castidad conyugal» (Gaudium et 
				spes,SI). 	. 
			4 Audiencia general; l-VIII-1984. 
		
		\subsection{Los conyuges deben regirse por la conciencia, ajustada a la ley divina,
		 siempre dociles al Magisterio de la Iglesia, que la interpreta autenticamente}
		 	2. Antes del pasaje citado (cfr Gaudium et spes, 50), 
			el Concilio ensefia que los conyuges «con responsabilidad 
			humana y cristiana cumpliran su mision y, con docil re- 
			verencia hacia Dios» (Ibid.). Lo cual quiere decir que: «De 
			comun acuerdo y cornun esfuerzo, se formaran un juicio 
			recto, atendiendo tanto a su propio bien personal como al 
			bien de los hijos, ya nacidos 0 todavia por venir, discer- 
			niendo las circunstancias de los tiempos y del estado de 
			la vida, tanto materiales como espirituales; y, finalmente, 
			teniendo en cuenta el bien de la comunidad familiar, de 
			la sociedad temporal y de la propia Iglesia» (Ibid.). 
			Alllegar a este punto siguen palabras particularmen- 
			te importantes para deterrninar, con mayor precision, el 
			caracter moral de la «paternidad y maternidad responsa- 
			bles». Leemos: «Este juicio, en ultimo termino, deben for- 
			marlo ante Dios los esposos personalmente» (Ibid.). 
			Y continuando: «En su modo de obrar, los esposos 
			cristianos sean conscientes de que no pueden proceder a 
			su antojo, sino que siempre deben regirse por la concien- 
			cia, la cual ha de ajustarse a la ley divina rnisrna, dociles 
			al Magisterio de la Iglesia, que interpreta autenticarnente 
			esa ley a la luz del Evangelio. Dicha ley divina muestra el 
			pleno sentido del amor conyugal, 10 protege e impulsa a 
			la perfeccion genuinamente humana del mismo» (Ibid.). 
			3. La Constitucion conciliar, limitandose a recordar 
			las premisas esenciales para una «paternidad y materni- 
			dad res pons ables», las pone de relieve de manera total- 
			mente univoca, precisando los elementos constitutivos de 
			semejante paternidad y maternidad, es decir: el juicio ma- 
			duro de la conciencia personal en su relacion con la ley 
			divina, autenticamente interpretada por el Magisterio de 
			la Iglesia. 
			4. La Enciclica Humanae vitae, basandose en las mis- 
			mas premisas, avanza algo mas, ofreciendo indicaciones 
			concretas. Ello se ye, sobre todo, en el modo de definir la 
			«paternidad responable» (Humanae vitae, 10). Pablo VI tra 
			ta de precisar este concepto, encareciendo los divers os as- 
			pectos y excluyendo, de anternano, su reduccion a uno de 
			los aspectos «parciales», como hacen quienes hablan, ex- 
			clusivamente, del control de la natalidad. En efecto, desde 
			el principio, Pablo VI se ve guiado, en su argumentacion, 
			por una concepcion integral del hombre (cfr Humanae vi- 
			tae, 7~ y del amor conyugal (cfr Humanae vitae, 8, 9).
			
		\subsection{Los c6nyuges deben ser fieles al plan divino. 
		La Paternidad responsable puede sugerir la reducci6n 
		o la procreaci6n de una familia numerosa segun norrnas de prudencia}
			5. Se puede hablar de responsabilidad en el ejercicio 
			de la funcion paterna y materna, bajo distintos aspectos. 
			Asi, escribe et: «En relacion a los procesos biologicos, pa- 
			ternidad res pons able significa conocimiento y respeto de 
			sus funciones; la inteligencia descubre, en el poder de dar 
			la vida, leyes biologic as que forman parte de la persona 
			humana» (Humanae vitae, 10). Cuando se trata, Iuego, de 
			la dimension psicologica de «las tendencias del instinto y 
			de las pasiones, la paternidad responsable comporta el do- 
			minio necesario que sobre aquellas han de ejercer la ra- 
			zon y la voluntad» (Ibid.).
			Supuestos los antedichos aspectos intra-personales y 
			afiadiendo a ellos «las condiciones economic as y sociales», 
			es necesario reconocer que «la paternidad responsable se 
			pone en practica, ya sea con la deliberacion ponderada y 
			generosa de tener una familia numerosa, ya sea con la de- 
			cision, tomada por graves motivos y en el respeto de la 
			ley moral, de evitar un nuevo nacimiento durante algun 
			tiempo 0 por tiempo indefinido» (Ibid.). 
			Se sigue de ello que en la concepcion de la «paterni- 
			dad res pons able» esta contenida la disposicion no sola- 
			mente a evitar «un nuevo nacimiento», sino tambien a ha- 
			cer crecer la familia segun los criterios de prudencia. 
			Bajo est a luz, des de la cual es necesario examinar y 
			decidir la cuestion de la «paternidad responsable», queda 
			siempre como central «el orden moral objetivo, estableci- 
			do por Dios, cuyo fiel interprete es la recta conciencia» 
			(Ibid.).
			6. Los esposos, dentro de este ambito, cumplen «ple- 
			namente sus deberes para con Dios, para consigo mismo, 
			para con la familia y la sociedad, en una justa [erarquia de 
			valores» (Ibid.). No se puede, por tanto, hablar aqui de 
			«proceder segun el propio antojo», Al contrario, los con- 
			yuges deben «conformar su conducta a la intencion crea- 
			dora de Dios» (Ibid.).
			Partiendo de este principio, la Encfclica fundamenta 
			su argumentacion sobre «la estructura intima del acto 
			conyugal» y sobre «la inseparable conexion entre los dos 
			significados del acto conyugal» (cfr Humanae vitae, 12); 
			todo 10 cual ha sido ya tratado anteriormente. El relativo 
			principio de la moral conyugal resulta ser, por 10 tanto, la 
			fidelidad al plan divino, manifestado en la «estructura in- 
			tima del acto conyugal» y en. «el inseparable nexo entre 
			los dos significados del acto conyugal».		
		
	{\centering
		\section{LO L\'ICITO Y LO IL\'ICITO EN LA REGULACI\'ON DE LOS NACIMIENTOS}
	}
		\subsection{Son moralmente llicitos el aborto, la estertltzaclcn 
		directa y to dos los medios anticonceptivos}
			1.' Hemos dicho anteriormente que el principio de la 
			moral conyugal, que la Iglesia ensefia (Concilio Vaticano 
			Il, Pablo VI), es el criterio de la fidelidad al plan divino. 
			De acuerdo con este principio, la Enciclica Humanae 
			vitae, . distingue rigurosamente entre 10 que constituye el 
			modo moralmente ilicito de la regulacion de los nacimien- 
			tos 0, con mayor precision, de la regulacion de la fertili- 
			dad, y el moralmente recto. 	. 
			En primer lugar, es moralmente ilicita «la interrupcion 
			directa del proceso generador ya iniciado» (eaborto») (Hu- 
			manae vitae, 14), la «esterilizacion directa» y «toda accion 
			que, 0 en prevision del acto conyugal, 0 en su realizacion, 
			o en el desarrollo de sus consecuencias naturales, se pro- 
			ponga, como fin 0 como medio, hacer imposible la pro- 
			creacion» (!bid), por tanto to dos los medios contraconcep- 
			tivos. Es por el contrario moralmente licito «el recurso a 
			los periodos injecundos» (Humanae vitae, 16): «Por consi- 
			guiente, si para espaciar los nacimientosexisten serios 
			5 Audiencia general; 8-VIII-1984. 
			motivos, derivados de las condiciones fisicas 0 psicologi- 
			cas de los conyuges, 0 de circunstancias exteriores, la Igle- 
			sia ensefia que entonces es licito ten er en cuenta los rit- 
			mos naturales inmanentes alas funciones generadoras 
			para usar del matrimonio solo en los periodos infecundos 
			y asi regular la natalidad sin of ender los principios mora- 
			les ... » (Ibid.). 
		
		\subsection{Es moralmente licito el recurso a los period os infecundos. 
		Hay una diferencia esencial entre estos y los anticonceptivos}
			2. La Enciclica subraya de modo particular que «en- 
			tre ambos casos existe una diferencia esencial» (Ibid), esto 
			es, una diferencia de naturaleza erica: «En el primero, los 
			conyuges se sirven legitimamente de una disposicion na- 
			tural; en el segundo, impiden el desarrollo de los procesos 
			naturales» (Ibid). 
			De ello se derivan dos acciones con clasificacion eti- 
			ea divers a, mas aun, incluso opuesta: la regulacion natu- 
			ral de la fertilidad es moralmente recta, la contracepcion 
			no es moralmente recta. Esta diferencia esencial entre las 
			dos acciones (modos de actuar) concierne a su intrinseca 
			calificacion erica, si bien mi predecesor Pablo VI afirma 
			que «tanto en uno como en otro caso, los conyuges estan 
			de acuerdo en la voluntad positiva de evitar la prole por 
			razones plausibles», e incluso escribe: «buscando la segu- 
			rid ad de que no se seguira» (Ibid.). En estas palabras el do- 
			cumento admite que, si bien tambien los que hacen uso 
			de las practicas anticonceptivas puedan estar inspirados 
			por «razones plausibles», sin embargo ello no cambia la 
			caliiicacion moral que se [unda en la estructura misma del 
			acto conyugal como tal. 
			3. Se podra observar, en este pun to, que los conyu- 
			ges que recurren a la regulacion natural de la fertilidad 
			podrian carecer de los razones valid as de que se ha ha-
			blado anteriormente: pero esto constituye un problema 
			erica aparte, dado que se trata del sentido moral de la «pa- 
			ternidad y maternidad responsables». 
			Suponiendo que las razones para decidir no procrear 
			sean moralmente rectas, queda el problem a moral del 
			modo de actuar en tal caso, y esto se expresa en un acto 
			que -segun la doctrina de la Iglesia transmitida en la En-. 
			ciclica- posee su intrinseca calificacion moral positiva 0 
			negativa. La primera, positiva, corresponde a la «natural» 
			regulacion de la fertilidad; la segunda, negativa, corres- 
			ponde a la «contracepcion artificial».
			
		\subsection{Contenido normative-pastoral de la Enciclica Humanae vitae}
			4. Toda la argumentacion precedente se resume en 
			la exposicion de la doctrina contenida en la Humanae vi- 
			tae, advirtiendo en ella el caracter normativo y al mismo 
			tiempo pastoral. En la dimension normativa se trata de 
			precisar y aclarar los principios morales del actuar; en la 
			dimension pastoral se trata sobre to do de ilustrar la posi- 
			bilidad de actuar segun estos principios (eposibilidad de la 
			observancia de la ley divina», Humanae vitae, 20). 
			5. La teologia del cuerpo no es tanto una teoria, cuan- 
			to mas bien una especifica, evangelica, cristiana pedago- 
			gia del cuerpo. Esto se deriva del caracter de la Biblia, y 
			sobre to do del Evangelio que, como mensaje salvifico, re- 
			vela la que es el verdadero bien del hombre; a fin de mo- 
			delar -a medida de este bien- la vida en la tierra, en la 
			perspectiva de la esperanza del mundo futuro. 
			La Enciclica Humanae vitae, siguiendo esta linea, res- 
			ponde a la cuestion sobre el verdadero bien del hombre 
			como persona, en cuanto varon y mujer.
	
	{\centering
		\section{LA DOCTRINA SOBRE LA TRANSMIS\'ON DE LA VIDA}
	}
	. El dominio de si mismo corresponde a la constituci6n 
de la persona. Los «rnedios artlflclales» hacen 
del hombre un objeto de manipulaci6n 
1. (CuM es la esencia de la doctrina de la Iglesia acer- 
ea de la transmision de la vida en la comunidad conyu- 
gal, de esa doctrina que nos han recordado la Constitu- 
cion pastoral del Concilio Gaudium et spes y la Enciclica 
Humanae vitae del Papa Pablo VI? 
El problem a esta en .mantener la relacion adecuada 
entre 10 que se define «dominio ... de las [uerzas de la na- 
turaleza» (Humanae vitae, 2), y el «dominio de si» (Huma- 
nae vitae, 21), indispensable a la persona humana. El hom- 
bre conternporaneo manifiesta la tendencia a transferir 
los rnetodos propios del primer arnbito a los del segundo. 
«El hombre ha llevado a cabo progresos estupendos en el 
dominio y en la organizacion racional de las fuerzas de la 
naturaleza -leemos en la Enciclica-, de modo que tien- 
de a extender ese dominio a su mismo ser global: al cuer- 
po, a la vida psiquica, a la vida social y hasta las leyes que 
regulan la transmision de la vida» (Humanae vitae, 2). 
6 Audiencia general; 22-VIII-1984. 
Esta extension de la esfera de los medios de «domi- 
nio ... de las fuerzas de la naturaleza» amenaza a la perso- 
na humana, para la cual el metodo del «dominio de si» es 
y sigue siendo especifico. Efectivamente, el dominio de si 
corresponde a la constitucion fundamental de la persona: 
es precisamente un rnetodo «natural». En cambio, la trans- 
ferencia de los «medios artificiales» rompe la dimension 
constitutiva de la persona, priva al hombre de la subjeti- 
vidad que le es propia y hace de el un objeto de manipula- 
cion. 
2. El cuerpo humano no es solo el campo de reaccio- 
nes de caracter sexual, sino que es, al mismo tiempo, el 
medio de expresion del hombre integral, de la persona, 
que se revela a si misma a traves del «lenguaje del cuer- 
po». Este «lenguaje» tiene un importante significado inter- 
personal, especialmente cuando se trata de las relaciones 
reciprocas entre el hombre y la mujer. Ademas, nuestros 
analisis precedentes muestran que en este caso el «lengua- 
je del cuerpo» debe expresar, a un nivel determinado, la 
verdad del sacramento. Efectivamente, al participar del 
eterno plan de amor (<<Sacramentum absconditum in 
Deo>>), el «lenguaje del cuerpo» 'se convierte como en un 
«profetismo del cuerpo». 
Se puede decir que la Enciclica Humanae vitae lleva 
alas ultimas consecuencias, no solo logicas y morales, sino 
tambien practicas y pastorales, esta verdad sob re el cuer- 
po humano en su masculinidad y feminidad. 
3. La unidad de los aspectos del problema -de la di- 
mension sacramental (0 sea, teologica) y de la pesonalis- 
tica- corresponde a la global «revelacion del cuerpo». De 
aqui se deriva tambien la conexion de la vision estricta- 
mente teologica con la etica, que nace de la «ley natural». 
En efecto, el sujeto de la ley natural es el hombre no 
solo en el aspecto «natural» de su existencia, sino jambien 
en la verdad integral de su subjetividad personal. El se nos 
manifiesta, en la Revelacion, como hombre y mujer, en su 
plena vocacion temporal y escatologica. Es llamado por 
Dios para ser testigo e interprete del eterno designio del 
amor, convirtiendose en ministro del sacramento que, 
«desde el principio», se constituye en el signa de la «union 
de la came». 
No es Iicito separar el sentido unitivo del procreador 
4. Como ministros de un sacramento que se realiza 
por medio del consentimiento y se perfecciona por la 
union conyugal, el hombre y la mujer estan llamados a ex- 
presar ese misterioso «lenguaje» de sus cuerpos en toda la 
verdad que les es propia. Por medio de los gestos y de las 
reacciones, por medio de to do el dinamismo, reciproca- 
mente condicionado, de la tension y del gozo -cuya fuen- 
te direct a es el cuerpo en su masculinidad y feminidad, el 
. cuerpo en su accion e interaccion-> a traves de to do esto 
«habla» el hombre, la persona. 
El hombre y la mujer con el «lenguaje del cuerpo» de- 
sarrollan ese dialogo que =-segun el Genesis 2, 24-25- eo- 
menzo el dia de la creacion, Y precisamente a nivel de este 
«lenguaje del cuerpo» -que es algo mas que la sola reac- 
tividad sexual y que, como autentico lenguaje de las per- 
sonas, esta sometido alas exigencias de la verdad, es de- 
cir a normas morales objetivas-, el hombre y la mujer se 
expresan reciprocamente a si mismos, del modo mas ple- 
no y mas profundo, en cuanto les es posible por la misma 
dimension somatic a de la masculinidad y feminidad: el 
hombre y la mujer se expresan a si mismos en la medida 
de toda la verdad de su persona. 
5. El hombre es persona precisamente porque es due- 
no de si y se domina a si. mismo. Efectivamente, en cuan- 
to que es duefio de si mismo puede «donarse» al otro. Y 
esta es una dimension =-dimension de la libertad del 
don- que se convierte en esencial y decisiva para ese 
«lenguaje del cuerpo», en el que el hombre y la mujer se
expresan reciprocamente en la union conyugal. Dado que 
esta comunion es cornunion de personas, el «lenguaje del 
cuerpo» debe juzgarse segun el criterio de la verdad. Pre- 
cisamente la Enciclica Humanae vitae presenta este crite- 
rio, como confirman los pasajes antes citados. 
6. Segun el criteria de esta verdad, que debe expre- 
sarse con el «lenguaje del cuerpo», el acto conyugal «sig- 
nifica» no solo el amor, sino tambien la fecundidad poten- 
cial, y por esto no puede ser privado de su pleno y ade- 
cuado significado procreador, porque uno y otro pertene- 
cen a la verdad intima del acto conyugal: uno se realiza 
juntamente con el otro y, en cierto sentido, el uno a tra- 
ves del otro. Asi ensefia la Enciclica (cfr Humanae vitae, 
12). Por 10 tanto, en este caso el acto conyugal, privado de 
su verdad interior, al ser privado artificialmente de su ca- 
pacidad procreadora, deja tambien de ser acto de amor. 
7. 'Puede decirse que en el caso de una separacion ar- 
tificial de estos dos significados,en el acto conyugal se 
realiza una union corpore a, pero no corresponde a la ver- 
dad interior ni a la dignidad de la comunion personal: 
communio personarum Efectivamente, est a comunion 
exige que el «lenguaje del cuerpo» se exprese reciproca- 
mente en la verdad integral de su significado. Si falta esta 
verdad, no se puede hablar ni de la verdad del dominio 
de si, ni de la verdad del don reciproco y de la reciproca 
aceptacion de si por parte de la persona. Esta violacion 
del orden interior de la cornunion conyuga!, que hunde 
sus rakes en el orden mismo de la persona, constituye el 
mal esencial del acto anticonceptivo.
	
	{\centering
		\section{LA REGULACI\'ON DE LA NATALIDAD SEG\'UN LA TRADICI\'ON}
	}
	La Humanae vitae aprueba plenamente la regulacion 
natural de la fertilidad 
1. La Enciclica Humanae vitae, demostrando el mal 
moral de hi anticoncepcion, al mismo tiempo, aprueba pie- 
namente la regulaci6n natural de la natalidad y, en este 
sentido, aprueba la paternidad y maternidad responsables. 
Hay que excluir aqui que pueda ser calificada de «respon- 
sable», desde el punto de vista etico, la procreacion en la 
que se recurre a la anticoncepcion para realizar la reg~- 
lacion de la natalidad. El verdadero concepto de «paterm- 
dad y maternidad responsables», por el contrario, esta uni- 
do a la regulacion de la natalidad honesta desde el punto 
de vista etico, 
Una regulacion etlcamente honest a exige convicciones 
sob re los verdaderos valores de la vida y de la familia 
2. Leemos a este proposito: «Una practica honesta de 
la regulacion de la natalidad exige sobre todo a los espo- 
7 Audiencia general, 29- VIIl-1984.
:iOS adquirir y poseer solidas convicciones sobre los ver- 
daderos valores de la vida y de la familia, y tambien una 
tendencia a procurarse un perfecto dominio de si mismos. 
El dominio del instinto, mediante la razon y la voluntad li- 
bre, irnpone, sin ningun genero de duda, una ascetica, 
para la que las manifestaciones afectivas de la vida con- 
yugal est en en conformidad con el.orden recto y particu- 
larmente para observar la continencia periodica. Esta dis- 
ciplina, propia de la pureza de los esposos, lejos de perju- 
dicar el am or conyugal, le confiere un valor humano mas 
sublime. Exige un esfuerzo continuo, pero, en virtud de 
su influjo beneficioso, los conyuges desarrollan integra- 
mente su personalidad, enriqueciendose de valores espi- 
rituales ... » tHumanae vitae, 21). 
3. La Enciclica ilustra luego las consecuencias de este 
comportarniento no solo para los mismos esposos, sino 
tambieh para toda la familia, entendida como comunidad 
de personas. Habra que volver a tomar en consideracion 
este tema. La Enciclica subraya que la regulacion de la na- 
talidad eticamente honesta exige de los conyuges ante 
todo. un determinado comportamiento familiar y procrea- 
dor. esto es, exige a los esposos «adquirir y poseer solidas 
convicciones sobre los verdaderos valores de la vida y de 
la familia» (lbid). Partiendo de esta premisa, ha sido ne- 
cesario proceder a una consideracion global de la cues- 
tion, como hizo el Sinodo de los Obispos del ana 1980 (<<De 
muneribus familiae christianae>>). Luego, la doctrina rela- 
tiva a este problema particular de la moral conyugal y fa- 
miliar, de que trata la Enciclica Humanae vitae, ha encon- 
trado su justo puesto y la optica oportuna en el ccintexto 
total de la Exhortacion Apostolica Familiaris consortia. La 
teologia del cuerpo, sobre todo como pedagogia del cuer- 
po, hunde sus raices, en cierto sentido, en la teologia de la 
familia y, a la vez, lleva a ella. Esta pedagogia del cuerpo, 
cuya clave es hoy la Enciclica Humanae vitae, solo se ex- 
plica en el contexto pleno de una vision correcta de los va- 
	lores de la vida y de la familia.
La doctrina sobre la pureza como vida del espiritu 
4. En el texto antes citado el Papa Pablo VI se remite 
a la castidad conyugal, al escribir que la observancia de 
la continencia periodica es la forma de dominio de SI, don- 
de se manifiesta «la pureza de los esposos- (Ibid.). 
Al emprender ahora un analisis mas pro fun do de este 
problema, hay que tener presente toda la doctrina sobre 
la pureza, entendida como vida del espiritu (cfrGals, 25), 
que ya hemos considerado anteriorrnente, a fin de com- 
prender asi las respectivas indicaciones de la Enciclica so- 
bre el tema de la «continencia periodical). Efectivamente, 
esa doctrina sigue siendo la verdadera razon, a partir de 
la cualla ensefianza de Pablo VI define la regulacion de 
la natalidad y la paternidad y maternidad responsables co- 
ma eticamente honestas. 
Aunque la «periodicidad. de la con tin en cia se aplique 
en este caso a los llamados «ritmos naturales» (Humanae 
vitae, 16), sin embargo, la continencia misma es una de- 
terminada y permanente actitud moral, es virtud, y por 
esto, to do el modo de cornportarse, guiado por ella, ad- 
quiere caracter virtuoso. La Enciclica subraya bastante 
claramente que aqui no se trata solo de una determina- 
da «tecnica», sino de la etica en el sentido estricto de la 
palabra como moralidad de un comportamiento. 
Por tanto, la Enciclica pone de relieve oportunamen- 
te, por un lado, la necesidad de res pe tar en tal comporta- 
mien to el orden establecido por el Creador, y, por otro, la 
necesidad de la motivacion inmediata de caracter etico, 
5. Respecto al primer aspecto leemos: «Usufructuar 
( ... ) el don dei ainor conyugal respetando las leyes del pro- 
ceso generador significa reconocerse, no arbitros de las 
fuentes de la vida humana, sino mas bien administrado- 
res del plan 'establecido por el Creador» (Humanae vitae, 
13). «La vida humana es sagrada» -como recordo nues- 
tro predecesor de s. m. Juan XXIII en la Enciclica Mater 
et Magistra-, «desde su comienzo compromete directa-
mente la acci6n creadora de Dios» (AAS 53, 1961; cfr Hu- 
manae vitae, 13). En cuanto a la motivacion inmediata, la 
Enciclica «Humanae vitae» exige que «para espaciar los 
nacimientos existan serios motivos, derivadosde las con- 
diciones fisicas 0 psicol6gicas de los c6nyuges 0 de cir- 
cunstancias exteriores ... » (Humanae vitae, 16). 
. 6. En el caso de una regulaci6n moralmente recta de 
la natalidad que se realiza mediante la continencia peri6- 
dica, se trata claramente de practicar la castidad conyu- 
gal, es decir, de una determinada actitud etica. En ellen- 
guaje biblico diriamos que se trata de vivir del espiritu (cfr 
GaI5,25). 
La regulaci6n moralmente recta se denomina tam- 
bien «regulaci6n natural de la natalidad», 10 que puede ex- 
plicarse coma conformidad con la «ley natural». Por «ley 
natural» entendemos aqui el «orden de la naturaleza» en 
el campo de la procreaci6n, en cuanto es comprendido 
por la recta raz6n: este orden es la expresi6n del plan del 
Creador sobre el hombre. Y esto precisamente es 10 que 
la Enciclica, juntamente con toda la Tradici6n de la doe- 
trina y de la practica cristiana, subraya de modo especial: 
el caracter virtuoso de la actitud que se manifiesta con la 
regulaci6n «natural» de la natalidad, esta determinado no 
tanto por la fidelidad a una impersonal «ley natural», cuan- 
to al Creador-persona, fuente y Senor del orden que se ma- 
nifiesta en esta ley. 
Desde este punto de vista, la reducci6n a la sola regu- 
laridad biol6gica, separada del «orden de la naturaleza», 
esto es, del «plan del Creador», deform a el autentico pen- 
samiento de la Enciclica Humanae vitae (cfr Humanae vi- 
tae, 14). 	. 
El documento presupone ciertamente est a regulari- 
dad biologica, como la expresion del «orden de la natura- 
leza» esto es, del plan providencial del Creador.
	
	{\centering
		\section{S\'OLO CON RAZONES JUSTIFICADAS SE PUEDEN REGULAR LOS NACIMIENTOS}
	}
	La Humanae vitae habla de la regulacion 
se gun los ritmos naturales impresos 
en la naturaleza humana 	. 
1. Hemos hablado anteriorrnente de la regulacion 
honesta de la fertilidad segun la doctnna contenida en ~<;t 
Enciclica Humanae vitae (n. 19) y en la Exhortacion Fami- 
liaris consortio. La cualificaci6n de «natural», qu.e. se atr:- 
buye ala regulaci6n moralmente recta de la Iertilidad (si- 
guiendo los ritmos naturales, cfr ll.umanae VItae, 16), se 
explica con el hecho de que el relativo modo de cornpor- 
tarse corresponde a la verdad ~e ~a persona y, consiguien- 
temente, a su dignidad: una dignidad que P?r naturaleza 
afecta al hombre en cuanto ser racional y hbre. El hom- 
bre coma ser racional y libre, puede y debe releer con 
per~picacia el ritmo biol6gico qu~ pertene~e al orden na- 
tural.Puede y debe adecuarse a el para ejercer esta «pa- 
ternidad-maternidad» responsable que, de acuerdo c~ ;1 
designio del Creador, esta inscrita en el orden ~atur e 
la fecundidad humana. El concepto de regulacion moral- 
mente recta de la fertilidad no es smo la relectura del «len- 
8 Audiencia general, 5-IX-1984. 
guaje del cuerpo» en la verdad. Los mismos «ritrnos natu- 
rales inrhanentes en las funciones generadoras» pertene- 
cen a la verdad objetiva dellenguaje que las personas in- 
teresadas deberian releer en su contenido objetivo pleno. 
Hay que tener presente que el «cuerpo habla» no s610 con 
toda la expresi6n externa de la masculinidad y feminidad, 
sino tambien con las estructuras internas del organismo, 
de la reactividad somatica y psicosomatica. Todo ello debe 
tener ellugar que le corresponde en ellenguaje con que 
dialogan los c6nyuges en cuanto personas llamadas a la 
comuni6n en la «union del cuerpo». 
2. Todos los esfuerzos ten dentes al conocimiento 
cad a vez rnas precise de los «ritmos naturales» que se ma- 
nifiestan en relaci6n con la procreaci6n humana, todos los 
esfuerzos tambien de los consultorios farniliares y, en fin, 
de los mismos c6nyuges interesados, no miran a «biologi- 
zar» el-Ienguaje.del cuerpo (a «biologizar la etica», coma 
algunos opinan erroneamente), sino exclusivamente a ga- 
rantizar la verdad integral a ese «lenguaje del cuerpo» con 
el que los c6nyuges deben expresarsecon madurez fren- 
te alas exigencias de la paternidad y maternidad respon- 
sables. 
La Enciclica Humanae vitae subraya en varias ocasio- 
nes que la «paternidad res pons able» esta vinculada a un - 
esfuerzo y tes6n continuos, y que se lleva a efecto al pre- 
cio de una ascesis concreta (cfr Humanae vitae, 21). Estas 
y otras expresiones semejantes hacer ver que en el C'1S0 
de la «paternidad res pons able», 0 sea de la regulacion de 
la fertilidad moralmente recta, se trata de 10 que es el bien 
verdadero de las personas humanas y de 10 que correspon- 
de a la verdadera dignidad de la persona. 
Sin serios motivos se convertiria en fuente de ab us os 
3. El recurso a los «periodos infecund os» en la convi- 
vencia conyugal puede ser fuente de ab us os si los conyu- 
ges tratan asi de eludir si!l ra~one~ justificadas la procrea- 
ci6n rebajandola a un mvel mfenor al que es moralmen- 
te j~sto, de los nacimientos en su familia. Es precise q,ue 
se establezca este nivel justo teniendo en cuenta r;o. ~olo 
el bien de la propia familia y estado de. salud ~ posibilida- 
des de los mismos c6nyuges, sino tambien el bien de la so- 
ciedad a que pertenecen, de la Iglesia y hasta de la huma- 
nidad entera. 
La Enciclica Humanae vitae presenta la «paternidad 
res pons able» coma expresi6n de un alto valor eti<:o .. De 
ningun modo va enderezada unilateralmente a la Iimita- 
ci6n y, menos aim, a la exclusi6n de la prole;,supone tam- 
bien la disponibilidad a acoger una prole .mas numerose;. 
Sobre to do, segun la Enclica Humanae vitae, la «paterm- 
dad res pons able» realiza «una vinculaci6n m~s profunda 
con el orden moral objetivo establecido por DI~S, cuyo fiel 
interprete es la recta conciencia» (Humanae vitae, 10). .. 
4. La verdad de la paternidad responsable y su reali- 
zaci6n va unida a la madurez moral de la persona, y es 
aqui.donde muy frecuenteme~t~ se manifesta la 1iyergen- 
cia entre aquello a que la Enciclica atnbuye e~plicltamen- 
te el primado y aquello a 10 que se da este pnmado en la 
mentalidad corriente. 
En la Enciclica se pone en primer plano la ~imensi6n 
etica del problema subrayando el papel,de l~ virtud de ~a 
templanza rectamente entendida. En el ambito de esta di- 
mensi6n hay tambien un «metodo» adecuado para actuar 
segun el. En el modo corriente de pensar.acontec~ con ~r,e- 
cuencia que el «metodo», desvinculado de la dimension 
etica que le es propia, s~ po~e en acto de modo ~eramen- 
te funcional y hasta utilitano. Separando el «metodo na- 
tural» de la dimensi6n etica, se deja, de percibir le; diferen- 
cia existente entre este y otros «metodos» (medios artifi- 
ciales) y se llega a hablar de el coma si se tratase s610 de 
una forma diversa de anticoncepci6n. 
5. Desde el punto de vista de la autentica doctrina ex- 
presada en la Enciclica Humanae vitae, es importante, por 
'on iguiente, presentar correctamente el rnetodo a que 
alude dicho documento (cfr Humanae vitae, 16); es imp or- 
tante sobre todo projundizar en la dimension etica, en 
cuyo arnbito el rnetodo por ser «natural» asume el signi- 
ficado de metodo honesto «rnoralmente recto». Y, por ello, 
en el marco de este analisis nos convendra dedicar la aten- 
ci6n principalmente a 10 que afirma la Enciclica sobre el 
tema del dominio de si mismo y sobre la continencia: Sin 
una interpretaci6n penetrante de este tema no llegaremos 
al nucleo de la verdad moral ni tampoco al nucleo de la 
verdad antropo16gica del problema. Ya se ha hecho notar 
anteriormente que las raices de este problem a se hunden 
en la teologia del cuerpo: es esta (cuando pasa a ser, coma 
debe, pedagogia del cuerpo) la que constituye en realidad 
el «metodo» moralmente honesto de la regulaci6n de la 
natalidad entendido en su sentido mas profundo y mas 
pleno. ' 
6. Expresando a continuaci6n los caracteres deIos 
valores especificamente morales de la regulacion «natu- 
ral» de la natalidad (es decir, honesta, 0 sea moralmente 
recta), el autor aporta a la vida familiar frutos de sereni- 
dad y de paz, y facilita la soluci6n de otros problemas: fa- 
vorece la atenci6n hacia el otro conyuge: ayuda a supe- 
rar el egoismo, enemigo del verdadero amor, y enraiza 
mas su sentido de responsabilidad. Los padres adquieren 
asi la capacidad de un influjo mas profundo y eficaz para 
educar a los hijos; los nifios y los j6venes crecen en la jus- 
ta est:ima de los valores humanos y en el desarrollo sere- 
no y armonico de sus facultades espirituales y sensibles» 
(Humanae vitae, 21). 
7. Las frases citadas completan el cuadro de 10 que 
la Enciclica Humanae vitae entiende por «practica hones- 
ta de la regulaci6n de la natalidad» (!bid.).
	
	{\centering
		\section{LA VOCACI\'ON MATRIMONIAL}
	}
	La vocaclon cristiana iniciada en el Bautismo se 
especifica y fortalece con el sacramento del Matrimonio 
1. Refiriendonos a la doctrina contenida en ~a Enci- 
clica Humanae vitae, trataremos de delinear ultenormen- 
	te la vida espiritual de los esposos. 	. . 
Estas son las grandes palabras de l~ En~lch.ca: «La 
Iglesia, al mismo tiempo que ens~fla las exigencies impres- 
cindibles de la ley divina, anuncia la salvacion y ab re con 
los sacramentos los caminos de la gracia, la 'cual hace del 
hombre una nueva criatura, capaz de corresponder en el 
amor y la verdadera libertad al designio de su Creadory 
Salvador y de encontrar suave el yugo de Cristo. 
»Los esposos cristi~?os, I?u~s, d6.ci~e~ a su voz, deben 
recordar que su vocacion cnstla~a, iniciada en el bautis- 
rno, se ha especificado y f<;>rtalecldo ulteriorrnente con el 
sacramenta del matrimomo. Par la mtsmo, los c01?yu~es 
son corroborados y coma consagrados para cump~lr fiel- 
mente los propios deberes, para re~lizar. su Voc~clon has- 
ta la perfecci6n y para dar un testirnomo pr,?plO de .e~l?s 
delante del mundo. A ellos ha confiado el ~enor la mlslo~ 
de hacer visible ante los hombres la santidad y la suavi- 
9 Audiencia general, 3-X-1984.
dad de la ley que une el amor mutuo de los esposos con 
su cooperaci6n al am or de Dios, autor de la vida huma- 
na» (Humanae vitae, 25). 
La denuncia del mal moral del acto antlconceptivo 
y la normas sobre la paternidad responsable, 
son parte de la vocacion que deben vivir los esposos 
2. Al mostrar el mal moral del acto anticonceptivo, y 
delineado, al mismo tiempo, un cuadro posiblemente in- 
tegral de la practica «honesta» de la regulaci6n de la fer- 
tilidad, 0 sea, de la paternidad y maternidad res pons abies, 
la Enciclica Humanae vitae crea las premisas que permi- 
ten trazar las grandes Iineas de la espiritualidad cristiana 
'de la vocaci6n y de la vida conyugal e, igualmente, de la 
de los padres y de la familia. 
Mas aim, puede decirse que la Enciclica presupone 
toda la tradici6n de esta espiritualidad, que hunde sus rai- 
ces en las fuentes biblicas, ya analizadas anteriormente, 
brindando la ocasi6n de reflexionar de nuevo sobre ell as 
y hacer una sintesis adecuada. 
Con vie ne recordar aqui 10 que se ha dicho sobre la re- 
laci6n organic a entre la teologia del cuerpo y la pedago- 
gia del cuerpo. Esta «teologia-pedagogia», en efecto, cons- 
tituye ya de por si el nucleo esencial de la espiritualidad 
conyugal. Y esto 10 indican tambien las frases de la Enci- 
clica que hemos citado. 
3. Ciertamente, releeria e interpretaria de forma erro- 
nea la Enciclica Humanae vitae el que viese en ella tan 
s610 la reduccion de la «pat ern id ad y maternidad respon- 
sables» a los solos «ritrnos biol6gicos de la fecundidad». El 
autor de la Enciclica desaprueba energicamente y contra- 
dice toda forma de interpretaci6n reductiva (yen este sen- 
tido «parcial»), y vuelve a prop on er con insist en cia la com- 
prensi6n integral. La paternidad-maternidad responsable, 
entendida intcgralinrtn». 1111 I 1111. que un imP9r:tan'll.' 
elemento de toda /(1 " I'll ill/l/l/rlllrI cot! IIgClI Y [amiliar, cs 
decir, de esa v "I~ luu tll' I •• <illl' hul la ,I l xto citad cl' 
la Humanae vitae nyuges dcbcn 
realizar «su v en:j III II \. I I III P 'I'I' -cci n» (Ibid.). El sacra- 
mento del matrlmuulu 10. ('()IT 	ra Y como consagra 
para conseguirla ( ,I'I' IMd.). 	, . 
A la luz de la lortriun, 'XI r .sada en la EnClcilca, con- 
viene que no dcm s IlItI r .ucnta de es~, «[uerza ,cor~o- 
borante» que est[, 1Il1i la la « onsagracron SUl genens» 
del sacrarnento d 'In atrimonio. 
Puesto qll el anali is cl la problematica etica del do- 
cumento de Pablo VI e taba centrado sobre to do en la 
exactitud de la respectiva norma, el esbozo de la espiri- 
tualidad conyugal que alii se encuentra, intenta po~er de 
relieve precisamente estas «fuerzas» que hac en posible el 
autentico testimonio cristiano de la vida conyugal. 
El lIevar una recta conduct a moral 
comporta dificultades y sacrificios 
4. «No es nuestra intenci6n ocultar las dificultades, a 
veces graves, inherentes a la vida de los c6nyuges cristia- 
nos; para ellos, coma para todos, la puerta es estrecha y 
angosta la senda que lleva a la vida (cfr Mt 7! 14). ?ero la 
esperanza de esta vida debe iluminar su cammo m~en~ras 
se esfuerzan animosamente por vivir con prudencia, JUS- 
ticia y piedad en el tiempo presente, conscientes .de que la 
forma de este mundo es pasajera» (Humanae vitae, ~5). 
En la Enciclica, la visi6n de la vida conyugal esta, en 
cada pasaje, marcada por el realismo cristiano, y esto es 
precisamente 10 que mas ayuda ~ c0l!-segUlr esas <~fuer- 
zas» que permiten formar la espmtualidad de l~s conyu- 
ges y de los padres en el espiritu de una autentica peda- 
gogia del coraz6n y del cuerpo. 
La misma conciencia «de la vida futura» abre, por de- 
cirlo asi, un amplio horizonte de esas [uerzas que deben 
guiarlos por la senda angosta (cfr Ibid) y conducirlos por 
la puerta estrecha (cfr Ibid) de la vocacion evangelica. 
La Enciclica dice: «Afronten, pues, los esposos los ne- 
cesarios esfuerzos, apoyados por la fe y por la esperanza, 
que no engafia, porque el am or de Dios ha sido difundido 
en nuestros corazones junto con el Espiritu Santo, que nos 
ha sido dado» (Ibid). 
Medios para vivir la santidad en el Matrimonio 
5. He aqui la «fuerza» esencial y fundamental: el amor 
injertado en el corazon (edifundido en los corazones») por 
el Espiritu San to. Luego la Enciclica indica corno los con- 
yuges deben implorar est a «fuerza» esencial y toda otra 
«ayuda divina» con la oracion; como deben obtener la gra- 
cia y el am or de la fuente siempre viva de la Eucaristia; 
corno deben superar «con humilde perseverancia» las pro- 
pias faltas y los propios pecados en el sacramenta de la Pe- 
nitencia. 
Estos son los medios -infatigables e indispensables- 
para formar la espiritualidad cristiana de la vida conyu- 
gal y familiar. Con elIos esa esencial y espiritualmente 
creativa «[uerza» de amor llega a los corazones humanos 
y, al mismo tiempo, a los cuerpos humanos en su subjeti- 
va masculinidad y feminidad. Efectivamente, este amor 
permite construir toda la convivencia de los esposos se- 
gun la «verdad del signo», por medio de la cual se cons- 
truye el matrimonio en su dignidad sacramental, como 
pone de relieve el punto central de la Enciclica (cfr Hu- 
manae vitae, 12).
	
	{\centering
		\section{LA ESPIRITUALIDAD CONYUGAL Y LOS FINES DEL MATRIMONIO}
	}
	El amor, fuerza para participar en el misterio 
de la creacion y de la red en cion 
1. Continuamos delineando la espiritualidad conyu- 
	gal a la luz de la Enciclica Humanae vuae. 	. 
Segun la doctrina contenida en ella, e~ ~?nfOrmldad 
con las fuentes biblicas y con toda la Tradicion, el amo.r 
es -desde el punto de vista s.ubjetivo- «[uerza», ~s. decir, 
cap acid ad del espirjtu humane, de ani t r «t ologico» (0 
mejor «teologal»), Esta es, pu ,la [uerza qu~. e le ~a al 
homb~e para participar n 1 rnor n qu l~~ rrnsmo 
ama en el misterio de la rea i6n.dy dd' lea] r Cd ~c;o~). ~Sst~ 
	amor que «se compla ' n I v r 	. )~ 	or,., 
es, en el cual se expre a la I '~rfa 'spmlual (~l «frui. agus- 
tiniano) de todo valor ut ·n.ll .. '. z. ID jante al tOZO 
del mismo Creador, qu al pnl Ipl VI qu «era muy ue- 
no» (Gen 1,31). 
Si las [uerzas de la con .upis . in iia inten~an separar 
	el «lenguaje del cuerpo» cl ' la v • r dad. 	d err, tratan de 
10 Audiencia general. IO-X-' 84. 
[alsilicarlo, en cambio, la [uerza del amor 10 corrobora 
i mpre de ~I!evo en esa verdad, a fin de que el misterio 
de la redencion del cuerpo pueda fructificar en ella. 
El amor salvaguarda la unidad indisoluble 
de los significados del acto conyugal 
2. El mismo amor, que hace posible y hace ciertamen- 
te que el dialogo conyugal se realice segun la verdad ple- 
na ~e la vida de los esposos, es, a la vez, [uerza, 0 sea, ea- 
pacidad de caracter moral, orientada activamente hacia la 
plenitud del bien y, por esto mismo, hacia to do verdadero 
bien .. P~r ~~ cual, su tarea consiste en salvaguardar la uni- 
dad indivisible de los «~o~ significados del acto conyugal», 
de los que trata la Enciclica (Humanae vitae, 12), es decir, 
en proteger tanto el valor de la verdadera uni6n de los es- 
posos (esto es, de la comuni6n personal), coma el de la pa- 
ternidad y rnaternidad responsables (en su forma madu- 
	ra y digna del hombre). 	. 
Queda confirm ad a la ensefianza tradicional 
sobre la jerarquia de los fines del Matrimonio 
3. Se~un ellen~uaje tradicional, el amor, coma «fuer- 
z~» supenor, coordina las acciones de la persona, del ma- 
rido y de la mujer, en el ambito de los fines del matrimo- 
nto. Aunque ni la Constituci6n conciliar, ni la Enciclica, al 
afrontar el te!lla,· empleen el lenguaje acostumbrado en 
~tro nempo, sm embargo, tratan de aquello a 10 que se re- 
. fieren las expresiones tradicionales. 
	. 	El ~mor, con;o fuerza superior que el hombre y la mu- \ 
jer r~,Clben de DlOS, juntamente con la particular «consa- 
graciom ~:I sacramento del matrimonio, comporta una 
coordinacion correcta de los fines, segun los cuales -en 
la ensefianza tradicional de la Iglesia- se constituye el or- 
den moral (0 mejor, «teologal moral» de la vida de los es- 
posos. 
La doctrina de la Constituci6n Gaudium et spes, igual 
que la de la Enciclica Humanae vitae, clarifican el mismo 
Ol-den moral con referencia al amor, entendido coma 
fuerza superior que confiere adecuado contenido y valor 
a los actos conyugales segun la verdad de los dos signi- 
ficados, el unitivo y el procreador, respetando su indivisi- 
bilidad. 
Con este renovado planteamiento, la ensefianza tradi- 
cional sobre los fines del matrimonio (y sobre su jerar- 
quia) queda confirm ad a y a la vez se profundiza desde el 
punto de vista de la vida interior de los esposos, o sea, de 
la espiritualidad conyugal y familiar. 
La fuerza del amor conyugal puede 
con las dificultades que present a la concuplscencla: 
no hay confllcto de valores
flexion), para demostrar que en este caso no hay que ha- 
blar de «contradiccion», sino solo de «diiicultad». Ahora 
bien, la Enciclica misma subraya esta «dificultad» en va- 
rios pasajes. 
Y esta se deriva del hecho de que la [uerza del amor 
esta injertada en el hombre insidiado par la concupiscen- 
cia: en los sujetos humanos el amor choca con la triple con- 
cupiscencia (1 foh 2, 16), en particular con la concupiscen- 
cia de la came, que deforma la verdad del «lenguaje del 
cuerpo». Y, por esto, tampoco el amor esta en disposici6n 
de realizarse en la verdad del «lenguaje del cuerpo», si no 
es mediante el dominio de la concupiscencia. 
El amor matrimonial est a por su naturaleza 
	unido a la castidad 	. 
-So Si el elemento clave de la espiritualidad de los es- 
posos y de los padres -esa «fuerza» esencial que los con- 
yuges deben sacar continuamente de la «consagracion» 
sacramental- es el amor, este amor, como se deduce del 
texto de la Enciclica (cfr Humanae vitae, 20), esta por su 
naturaleza unido con la castidad que se manifiesta como 
dominio de si, 0 sea, como continente en particular, como 
continencia peri6dica. En el lenguaje biblico, parece alu- 
dir a esto el autor de la Carta a 10s Efesios, cuando en su 
texto «clasico» exhorta a los esposos a estar «sujetos los 
unos a 105 otros en el temor de Cristo» (Eph 5,21). 
Puede decirse que la Enciclica Humanae vitae es pre- 
cisamente el desarrollo de esta verdad biblica sobre la es- 
piritualidad cristiana conyugal y familiar. Sin embargo, 
para hacerlo aun mas claro, es preciso un analisis mds 
. profundo de la virtud de la continencia y de su particular 
significado para la verdad del mutuo «Ienguaje del cuer- 
PO» en la convivencia conyugal e (indirectamente) en la 
amplia esfera de las relaciones reciprocas entre el hom- 
bre y la mujer. 
Emprenderemos este analisis en las sucesivas re- 
flexiones del miercoles, 
\end{document}