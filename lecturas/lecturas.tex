%%%%%%%%%%%%%%%%%%%%%%%%%%%%%%%%%%%%% C A B E C E R A %%%%%%%%%%%%%%%%%%%%%%%%%%%%%%%%%%
% Definimos el estilo del documento
\documentclass[12pt, letterpaper]{report}

% Utilizamos un paquete para gestionar acentos y e\~nes
\usepackage[latin1]{inputenc}
\usepackage[T1]{fontenc}
\usepackage{xcolor}
\usepackage{pifont}
% Utilizamos el paquete para gestionar imagenes jpg
\usepackage{graphicx}
\graphicspath{ {images/} }
% Letra capiral
\usepackage{lettrine}
\usepackage{enumitem}
\usepackage{lmodern}

% Definimos la zona de la pagina ocupada por el texto
\oddsidemargin -1.0cm
\headsep -1.0cm
\textwidth=18.5cm
\textheight=23cm

\setlength{\parskip}{\baselineskip}

%Empieza el documento %%%%%%%%%%%%%%%%%%%% P R I N C I P I O %%%%%%%%%%%%%%%%%%%%%%%%%%%%%%
\begin{document}

\begin{center}
\Large {\bfseries \textcolor{red}{TIEMPO ORDINARIO}}
\end{center}

\begin{center}
\Huge {\bfseries III DOMINGO DEL TIEMPO ORDINARIO}
\end{center}

\begin{center}
\Large {\bfseries \textcolor{red}{PRIMERA LECTURA}}
\end{center}

\begin{center}
\large {\bfseries \textit{ \textcolor{red}{Los ninivitas se convirtieron de su mala vida.}}}
\end{center}

\Large {\bfseries Lectura de la profec\'ia de Jon\'as \hspace{1cm} \textcolor{red}{3, 1-5. 10}}

\lettrine[lines=2]{\bfseries \textcolor{red}{E}}{}\Large n aquellos d\'ias, vino la palabra del Se\~nor sobre Jon\'as:\\
--<<Lev\'antate y vete a N\'inive, la gran ciudad, y pred\'icale el mensaje que te digo.>>--\\
Se levant\'o Jon\'as y fue a N\'inive, como mand\'o el Se\~nor. N\'inive era una gran ciudad, tres d\'ias hacían falta para recorrerla. Comenz\'o Jon\'as a entrar por la ciudad y camin\'o durante un d\'ia, proclamando:\\
--<<!`Dentro de cuarenta d\'ias N\'inive ser\'a destruida!>>--\\
Creyeron en Dios los ninivitas; proclamaron el ayuno y se vistieron de saco, grandes y peque\~nos.\\
Y vio Dios sus obras, su conversi\'on de la mala vida; se compadeci\'o y se arrepinti\'o Dios de la cat\'astrofe con que había amenazado a N\'inive, y no la ejecut\'o.\\

{\bfseries Palabra de Dios.}

\Large {\bfseries \textcolor{red}{Salmo responsorial \hspace{1cm} Salmo 24, 4-5ab. 6-7bc. 8-9 (R.: 4a)}}

\Large {\bfseries \textcolor{red}{R/.}} \hspace{1cm} Se\~nor, ens\'e\~name tus caminos.

{\bfseries \textcolor{red}{V/.}} \hspace{1cm} Se\~nor, ens\'e\~name tus caminos,\\
. \hspace{2.5cm} instr\'uyeme en tus sendas:\\
. \hspace{2.5cm} haz que camine con lealtad;\\
. \hspace{2.5cm} ens\'e\~name, porque t\'u eres mi Dios y Salvador.
\hspace{1cm} {\bfseries \textcolor{red}{R/.}}

{\bfseries \textcolor{red}{V/.}} \hspace{1cm} Recuerda, Se\~nor, que tu ternura\\
. \hspace{2.5cm} y tu misericordia son eternas;\\
. \hspace{2.5cm} acu\'erdate de m\'i con misericordia,\\
. \hspace{2.5cm} por tu bondad, Se\~nor.
\hspace{1cm} {\bfseries \textcolor{red}{R/.}}

{\bfseries \textcolor{red}{V/.}} \hspace{1cm} El Se\~nor es bueno y es recto,\\
. \hspace{2.5cm} y ense\~na el camino a los pecadores;\\
. \hspace{2.5cm} hace caminar a los humildes con rectitud,\\
. \hspace{2.5cm} ense\~na su camino a los humildes.
\hspace{1cm} {\bfseries \textcolor{red}{R/.}}

\begin{center}
\Large {\bfseries \textcolor{red}{SEGUNDA LECTURA}}
\end{center}

\begin{center}
\large {\bfseries \textit{ \textcolor{red}{La representaci\'on de este mundo se termina.}}}
\end{center}

\Large {\bfseries Lectura de la primera carta del ap\'ostol San Pablo a los Corintios \hspace{1cm} \textcolor{red}{7, 29-31}}

\lettrine[lines=2]{\bfseries \textcolor{red}{D}}{}\Large igo esto, hermanos: que el momento es apremiante.\\
Queda como soluci\'on que los que tienen mujer vivan como si no la tuvieran; los que lloran, como si no lloraran; los que est\'an alegres, como si no lo estuvieran; los que compran, como si no poseyeran; los que negocian en el mundo, como si no disfrutaran de \'el: porque la representaci\'on de este mundo se termina.

{\bfseries Palabra de Dios.}

\newpage

\begin{center}
\Large {\bfseries \textcolor{red}{Aleluya \hspace{1cm} Mc 1, 15}}\\
Est\'a cerca el reino de Dios:\\
convert\'ios y creced en el Evangelio.
\end{center}

\begin{center}
\Large {\bfseries \textcolor{red}{EVANGELIO}}
\end{center}

\begin{center}
\large {\bfseries \textit{ \textcolor{red}{Convert\'ios y creed en el Evangelio.}}}
\end{center}

\Huge \textcolor{red}{\ding{64}} \Large {\bfseries Lectura del santo Evangelio seg\'un San Marcos \hspace{1cm} \textcolor{red}{1, 14-20}}

\lettrine[lines=2]{\bfseries \textcolor{red}{C}}{}\Large uando arrestaron a Juan, Jes\'us se march\'o a Galilea a proclamar el Evangelio de Dios. Dec\'ia:\\
--<<Se ha cumplido el plazo, est\'a cerca el reino de Dios: convert\'ios y creed en el Evangelio.>>--\\
Pasando junto al lago de Galilea, vio a Sim\'on y a su hermano Andr\'es, que eran pescadores y estaban echando el copo en el lago.\\
Jes\'us les dijo:\\
--<<Venid conmigo y os har\'e pescadores de hombres.>>--\\
Inmediatamente dejaron las redes y lo siguieron.\\
Un poco m\'as adelante vio a Santiago, hijo de Zebedeo, y a su hermano Juan, que estaban en la barca repasando las redes. Los llam\'o, dejaron a su padre Zebedeo en la barca con los jornaleros y se marcharon con \'el.

{\bfseries Palabra del Se\~nor.}
% Termina el documento %%%%%%%%%%%%%%%%%%%%%%%%%%%%% F I N %%%%%%%%%%%%%%%%%%%%%%%%%%%%%%%%%%%%%%%%%%%%%%%%%%%%%%
\end{document}
