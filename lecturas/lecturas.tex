%%%%%%%%%%%%%%%%%%%%%%%%%%%%%%%%%%%%% C A B E C E R A %%%%%%%%%%%%%%%%%%%%%%%%%%%%%%%%%%
% Definimos el estilo del documento
\documentclass[12pt, letterpaper]{report}

% Utilizamos un paquete para gestionar acentos y e\~nes
\usepackage[latin1]{inputenc}
\usepackage[T1]{fontenc}
\usepackage{xcolor}
\usepackage{pifont}
% Utilizamos el paquete para gestionar imagenes jpg
\usepackage{graphicx}
\graphicspath{ {images/} }
% Letra capiral
\usepackage{lettrine}
\usepackage{enumitem}
\usepackage{lmodern}

% Definimos la zona de la pagina ocupada por el texto
\oddsidemargin -1.0cm
\headsep -1.0cm
\textwidth=18.5cm
\textheight=23cm

\setlength{\parskip}{\baselineskip}

%Empieza el documento %%%%%%%%%%%%%%%%%%%% P R I N C I P I O %%%%%%%%%%%%%%%%%%%%%%%%%%%%%%
\begin{document}

\begin{center}
\Large {\bfseries \textcolor{red}{TIEMPO DE NAVIDAD}}
\end{center}

\begin{center}
\Huge {\bfseries Sagrada Familia de Nazaret}
\end{center}

\begin{center}
\Large {\bfseries \textcolor{red}{PRIMERA LECTURA}}
\end{center}

\begin{center}
\large {\bfseries \textit{ \textcolor{red}{El que teme al Se\~nor honra a sus padres.}}}
\end{center}

\Large {\bfseries Lectura del libro del Eclesi\'astico \hspace{1cm} \textcolor{red}{3, 2-6. 12-14}}

\lettrine[lines=1]{\bfseries \textcolor{red}{D}}{}\Large ios hace al padre m\'as respetable que a los hijos\\
y afirma la autoridad de la madre sobre su prole.\\
El que honra a su padre exp\'ia sus pecados,\\
el que respeta a su madre acumula tesoros;\\
el que honra a su padre se alegrar\'a de sus hijos\\
y, cuando rece, ser\'a escuchado;\\
el que respeta a su padre tendr\'a larga vida,\\
al que honra a su madre el Se\~nor lo escucha.\\
Hijo m\'io, s\'e constante en honrar a tu padre,\\
no lo abandones mientras vivas;\\
aunque chochee, ten indulgencia,\\
no lo abochornes mientras vivas.\\
La limosna del padre no se olvidar\'a,\\
ser\'a tenida en cuenta para pagar tus pecados.

{\bfseries Palabra de Dios.}

\newpage

\Large {\bfseries \textcolor{red}{Salmo responsorial \hspace{1cm} Salmo 127, 1-2. 3. 4-5 (R.: cf. 1)}}

\Large {\bfseries \textcolor{red}{R/.}} \hspace{1cm} Dichosos los que temen al Se\~nor y siguen sus caminos.

{\bfseries \textcolor{red}{V/.}} \hspace{1cm} Dichoso el que teme al Se\~nor\\
. \hspace{2.5cm} y sigue sus caminos.\\
. \hspace{2.5cm} Comer\'as del fruto de tu trabajo\\
. \hspace{2.5cm} ser\'as dichoso, te ir\'a bien.
\hspace{1cm} {\bfseries \textcolor{red}{R/.}}

{\bfseries \textcolor{red}{V/.}} \hspace{1cm} Tu mujer, como parra fecunda,\\
. \hspace{2.5cm} en medio de tu casa;\\
. \hspace{2.5cm} tus hijos, como renuevos de olivo,\\
. \hspace{2.5cm} alrededor de tu mesa.
\hspace{1cm} {\bfseries \textcolor{red}{R/.}}

{\bfseries \textcolor{red}{V/.}} \hspace{1cm} \'Esta es la bendici\'on del hombre que teme al Se\~nor.\\
. \hspace{2.5cm} Que el Se\~nor te bendiga desde Si\'on,\\
. \hspace{2.5cm} que veas la prosperidad de Jerusal\'en\\
. \hspace{2.5cm} todos los d\'ias de tu vida.
\hspace{1cm} {\bfseries \textcolor{red}{R/.}}

\begin{center}
\Large {\bfseries \textcolor{red}{SEGUNDA LECTURA}}
\end{center}

\begin{center}
\large {\bfseries \textit{ \textcolor{red}{La vida de familia vivida en el Se\~nor.}}}
\end{center}

\Large {\bfseries Lectura de la carta del ap\'ostol San Pablo a los Colosenses \hspace{1cm} \textcolor{red}{3, 12-21}}

\lettrine[lines=1]{\bfseries \textcolor{red}{H}}{}\Large ermanos:\\
Como elegidos de Dios, santos y amados, vest\'ios de la misericordia entra\~nable, bondad, humildad, dulzura, comprensi\'on. Sobrellevaos mutuamente y perdonaos, cuando alguno tenga quejas contra otro. El Se\~nor os ha perdonado: haced vosotros lo mismo. Y por encima de todo esto, el amor, que es el ce\~nidor de la unidad consumada. Que la paz de Cristo act\'ue de \'arbitro en vuestro coraz\'on; a ella hab\'eis sido convocados, en un solo cuerpo. Y sed agradecidos.\\
La palabra de Cristo habite entre vosotros en toda su riqueza; ense\~naos unos a otros con toda sabidur\'ia; correg\'ios mutuamente. Cantad a Dios, dadle gracias de coraz\'on, con salmos, himnos y c\'anticos inspirados. Y, todo lo que de palabra o de obra realic\'eis, sea todo en nombre del Se\~nor Jes\'us, dando gracias a Dios Padre por medio de \'el. \\
Mujeres, vivid bajo la autoridad de vuestros maridos, como conviene en el Se\~nor. Maridos, amad a vuestras mujeres, y no se\'ais \'asperos con ellas. Hijos, obedeced a vuestros padres en todo, que eso le gusta al Se\~nor. Padres, no exasper\'eis a vuestros hijos, no sea que pierdan los \'animos.

{\bfseries Palabra de Dios.}


\begin{center}
\Large {\bfseries \textcolor{red}{Aleluya \hspace{1cm} Col 3, 15a. 16a}}\\
Que la paz de Cristo act\'ue de \'arbitro\\ 
en vuestro coraz\'on;\\
la palabra de Cristo habite entre vosotros\\
en toda su riqueza.
\end{center}

\newpage

\begin{center}
\Large {\bfseries \textcolor{red}{EVANGELIO}}
\end{center}

\begin{center}
\large {\bfseries \textit{ \textcolor{red}{El ni\~no iba creciendo y se llenaba de sabidur\'ia}}}
\end{center}

\Huge \textcolor{red}{\ding{64}} \Large {\bfseries Lectura del santo Evangelio seg\'un San Lucas \hspace{1cm} \textcolor{red}{2, 22-40}}

\lettrine[lines=1]{\bfseries \textcolor{red}{C}}{}\Large uando lleg\'o el tiempo de la purificaci\'on, seg\'un la ley de Mois\'es, los padres de Jes\'us lo llevaron a Jerusal\'en, para presentarlo al Se\~nor, de acuerdo con lo escrito en la ley del Se\~nor: <<Todo primog\'enito var\'on ser\'a consagrado al Se\~nor>>, y para entregar la oblaci\'on, como dice la ley del Se\~nor: <<Un par de t\'ortolas o dos pichones.>>\\
Viv\'ia entonces en Jerusal\'en un hombre llamado Sime\'on, hombre justo y piadoso, que aguardaba el consuelo de Israel; y el Esp\'iritu Santo moraba en \'el. Hab\'ia recibido un or\'aculo del Esp\'iritu Santo: que no ver\'ia la muerte antes de ver al Mes\'ias del Se\~nor. Impulsado por el Esp\'iritu, fue al templo.\\
Cuando entraban con el ni\~no Jes\'us sus padres para cumplir con \'el lo previsto por la ley, Sime\'on lo tom\'o en brazos y bendijo a Dios diciendo:

--<<Ahora, Se\~nor, seg\'un tu promesa,\\
puedes dejar a tu siervo irse en paz.\\
Porque mis ojos han visto a tu Salvador,\\
a quien has presentado ante todos los pueblos:\\
luz para alumbrar a las naciones\\
y gloria de tu pueblo Israel.>>--

Su padre y su madre estaban admirados por lo que se dec\'ia del ni\~no. Sime\'on los bendijo, diciendo a Mar\'ia, su madre:

--<<Mira, \'este est\'a puesto para que muchos en Israel caigan y se levanten; ser\'a como una bandera discutida: as\'i quedar\'a clara la actitud de muchos corazones. Y a ti, una espada te traspasar\'a el alma.>>--

Hab\'ia tambi\'en una profetisa, Ana, hija de Fanuel, de la tribu de Aser. Era una mujer muy anciana; de jovencita hab\'ia vivido siete a\~nos casada, y luego viuda hasta los ochenta y cuatro; no se apartaba del templo d\'ia y noche, sirviendo a Dios con ayunos y oraciones. Acerc\'andose en aquel momento, daba gracias a Dios y hablaba del ni\~no a todos los que aguardaban la liberaci\'on de Jerusal\'en.\\
Y cuando cumplieron todo lo que prescrib\'ia la ley del Se\~nor, se volvieron a Galilea, a su ciudad de Nazaret.\\
El ni\~no iba creciendo y robusteci\'endose, y se llenaba de sabidur\'ia; y la gracia de Dios lo acompa\~naba.

{\bfseries Palabra del Se\~nor.}
% Termina el documento %%%%%%%%%%%%%%%%%%%%%%%%%%%%% F I N %%%%%%%%%%%%%%%%%%%%%%%%%%%%%%%%%%%%%%%%%%%%%%%%%%%%%%
\end{document}
