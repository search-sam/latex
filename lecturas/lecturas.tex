%%%%%%%%%%%%%%%%%%%%%%%%%%%%%%%%%%%%% C A B E C E R A %%%%%%%%%%%%%%%%%%%%%%%%%%%%%%%%%%
% Definimos el estilo del documento
\documentclass[12pt, letterpaper]{report}

% Utilizamos un paquete para gestionar acentos y e\~nes
\usepackage[latin1]{inputenc}
\usepackage[T1]{fontenc}
\usepackage{xcolor}
\usepackage{pifont}
% Utilizamos el paquete para gestionar imagenes jpg
\usepackage{graphicx}
\graphicspath{ {images/} }
% Letra capiral
\usepackage{lettrine}
\usepackage{enumitem}
\usepackage{lmodern}

% Definimos la zona de la pagina ocupada por el texto
\oddsidemargin -1.0cm
\headsep -1.0cm
\textwidth=18.5cm
\textheight=23cm

\setlength{\parskip}{\baselineskip}

%Empieza el documento %%%%%%%%%%%%%%%%%%%% P R I N C I P I O %%%%%%%%%%%%%%%%%%%%%%%%%%%%%%
\begin{document}

\begin{center}
\Large {\bfseries \textcolor{red}{TIEMPO ORDINARIO}}
\end{center}

\begin{center}
\Huge {\bfseries DOMINGO XVIII DEL TIEMPO ORDINARIO}
\end{center}

\begin{center}
\Large {\bfseries \textcolor{red}{PRIMERA LECTURA}}
\end{center}

\begin{center}
\large {\bfseries \textit{ \textcolor{red}{Cuando el malvado se convierte de su maldad, salva su vida}}}
\end{center}

\Large {\bfseries Lectura del Profeta de Ezequiel \hspace{1cm} \textcolor{red}{18, 25-28}}

\lettrine[lines=1]{\bfseries \textcolor{red}{A}}{}\Large s\'i dice el Se\~nor:

<<Coment\'ais: ``No es justo el proceder del Se\~nor''.

Escuchad, casa de Israel: ?`Es injusto mi proceder?, ?`O no es vuestro proceder el que es injusto?

Cuando el justo se aparta de su justicia, comete la maldad y muere, muere por la maldad que cometi\'o.

Y cuando el malvado se convierte de la maldad que hizo y practica el derecho y la justicia, \'el mismo salva su vida. Si recapacita y se convierte de los delitos cometidos, ciertamente vivir\'a y no morir\'a.>>

{\bfseries Palabra de Dios.}

\newpage

\Large {\bfseries \textcolor{red}{Salmo responsorial \hspace{1cm} Salmo 24, 4bc-5. 6-7. 8-9 (R.: 6a)}}

\Large {\bfseries \textcolor{red}{R/.}} \hspace{1cm} Recuerda, Se\~nor, que tu misericordia es eterna.

{\bfseries \textcolor{red}{V/.}} \hspace{1cm} Se\~nor, ens\'e\~name tus caminos,\\
. \hspace{2.5cm} instr\'uyeme en tus sendas:\\
. \hspace{2.5cm} haz que camine con lealtad;\\
. \hspace{2.5cm} ens\'e\~name, porque t\'u eres mi Dios y Salvador,\\
. \hspace{2.5cm} y todo el d\'ia te estoy esperando.
\hspace{1cm} {\bfseries \textcolor{red}{R/.}}

{\bfseries \textcolor{red}{V/.}} \hspace{1cm} Recuerda, Se\~nor, que tu ternura\\
. \hspace{2.5cm} y tu misericordia son eternas;\\
. \hspace{2.5cm} no te acuerdes de los pecados\\
. \hspace{2.5cm} ni de las maldades de mi juventud;\\
. \hspace{2.5cm} acu\'erdate de m\'i con misericordia,\\
. \hspace{2.5cm} por tu bondad, Se\~nor.
\hspace{1cm} {\bfseries \textcolor{red}{R/.}}

{\bfseries \textcolor{red}{V/.}} \hspace{1cm} El Se\~nor es bueno y es recto,\\
. \hspace{2.5cm} y ense\~na el camino a los pecadores;\\
. \hspace{2.5cm} hace caminar a los humildes con rectitud,\\
. \hspace{2.5cm} ense\~na su camino a los humildes.
\hspace{1cm} {\bfseries \textcolor{red}{R/.}}

\newpage

\begin{center}
\Large {\bfseries \textcolor{red}{SEGUNDA LECTURA}}
\end{center}

\begin{center}
\large {\bfseries \textit{ \textcolor{red}{Tened entre vosotros los sentimientos propios de Cristo Jes\'us}}}
\end{center}

\Large {\bfseries Lectura de la carta del ap\'ostol San Pablo a los Filipenses \hspace{1cm} \textcolor{red}{2, 1-11}}

\lettrine[lines=1]{\bfseries \textcolor{red}{H}}{}\Large ermanos:\\
Si quer\'eis darme el consuelo de Cristo y aliviarme con vuestro amor, si nos une el mismo Esp\'iritu y ten\'eis entra\~nas compasivas, dadme esta gran alegr\'ia: manteneos un\'animes y concordes con un mismo amor y un mismo sentir.

No obr\'eis por envidia ni por ostentaci\'on, dejaos guiar por la humildad y considerad siempre superiores a los dem\'as. No os encerr\'eis en vuestros intereses, sino buscad todos el inter\'es de los dem\'as. Tened entre vosotros los sentimientos propios de una vida en Cristo Jes\'us.

El, a pesar de su condici\'on divina, no hizo alarde de su categor\'ia de Dios; al contrario, se despoj\'o de su rango y tom\'o la condici\'on de esclavo, pasando por uno de tantos.

Y as\'i, actuando como un hombre cualquiera, se rebaj\'o hasta someterse incluso a la muerte, y una muerte de cruz. Por eso Dios lo levant\'o sobre todo y le concedi\'o el <<Nombre-sobre-todo-nombre>>, de modo que al nombre de Jes\'us toda rodilla se doble en el Cielo, en la Tierra, en el Abismo y toda lengua proclame:

<<!`Jesucristo es Se\~nor!>> para gloria de Dios Padre.\\

{\bfseries Palabra de Dios.}


\begin{center}
\Large {\bfseries \textcolor{red}{Aleluya \hspace{1cm} Jn 10, 27}} \\
Mis ovejas escuchan mi voz\\
--dice el Se\~nor--,\\
y yo las conozco, y ellas me siguen.
\end{center}

\begin{center}
\Large {\bfseries \textcolor{red}{EVANGELIO}}
\end{center}

\begin{center}
\large {\bfseries \textit{ \textcolor{red}{Los publicanos y las prostitutas os llevan la delantera en el camino del Reino de Dios}}}
\end{center}

\Huge \textcolor{red}{\ding{64}} \Large {\bfseries Lectura del santo Evangelio seg\'un San Mateo \hspace{1cm} \textcolor{red}{21, 28-32}}

\lettrine[lines=1]{\bfseries \textcolor{red}{E}}{}\Large n aquel tiempo, dijo Jes\'us a los sumos sacerdotes y a los ancianos del pueblo:\\
--<<?`Qu\'e os parece? Un hombre ten\'ia dos hijos. Se acerc\'o al primero y le dijo: ``Hijo, ve hoy a trabajar en la vi\~na''. \'El le contest\'o: ``No quiero''. Pero despu\'es recapacit\'o y fue.\\
Se acerc\'o al segundo y le dijo lo mismo. \'El le contest\'o: ``Voy, se\~nor''. Pero no fue.\\
?`Qui\'en de los dos hizo lo que quer\'ia el padre?>>--\\
Contestaron:\\
--<<El primero.>>--\\
Jes\'us les dijo:\\
--<<Os aseguro que los publicanos y las prostitutas os llevan la delantera en el camino del reino de Dios. Porque vino Juan a vosotros ense\~n\'andoos el camino de la justicia, y no le cre\'isteis; en cambio, los publicanos y prostitutas le creyeron. Y, aun despu\'es de ver esto, vosotros no recapacitasteis ni le cre\'isteis.>>--

{\bfseries Palabra del Se\~nor.}
% Termina el documento %%%%%%%%%%%%%%%%%%%%%%%%%%%%% F I N %%%%%%%%%%%%%%%%%%%%%%%%%%%%%%%%%%%%%%%%%%%%%%%%%%%%%%
\end{document}
