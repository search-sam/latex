%%%%%%%%%%%%%%%%%%%%%%%%%%%%%%%%%%%%% C A B E C E R A %%%%%%%%%%%%%%%%%%%%%%%%%%%%%%%%%%
% Definimos el estilo del documento
\documentclass[12pt, letterpaper]{report}

% Utilizamos un paquete para gestionar acentos y e\~nes
\usepackage[latin1]{inputenc}
\usepackage[T1]{fontenc}
\usepackage{xcolor}
\usepackage{pifont}
% Utilizamos el paquete para gestionar imagenes jpg
\usepackage{graphicx}
\graphicspath{ {images/} }
% Letra capiral
\usepackage{lettrine}
\usepackage{enumitem}
\usepackage{lmodern}

% Definimos la zona de la pagina ocupada por el texto
\oddsidemargin -1.0cm
\headsep -1.0cm
\textwidth=18.5cm
\textheight=23cm

\setlength{\parskip}{\baselineskip}

%Empieza el documento %%%%%%%%%%%%%%%%%%%% P R I N C I P I O %%%%%%%%%%%%%%%%%%%%%%%%%%%%%%
\begin{document}

%\begin{center}
%\Large {\bfseries \textcolor{red}{LA SAGRADA FAMILIA DE JES\'US, MAR\'IA Y JOS\'E}}
%\end{center}

\begin{center}
\Huge {\bfseries RITO PARA RECONCILIAR A VARIOS PENITENTES CON
CONFESI\'ON Y ABSOLUCI\'ON INDIVIDUAL}
\end{center}

\Large {\bfseries \textcolor{red}{Canto}}

\large {\textcolor{red}{ Una vez reunidos los fieles, y mientras el sacerdote entra, si parece oportuno, se entona alg\'un salmo, ant\'ifona u otro canto adaptado a las circunstancias.}}

\begin{center}
\Huge {\bfseries Ritos Iniciales}
\end{center}

\Large {\bfseries \textcolor{red}{Saludo}}

\large {\textcolor{red}{ Terminado el canto, el sacerdote saluda a los asistentes, diciendo:}}

\noindent
\Large {\bfseries \textcolor{red}{R/.}} \hspace{1cm} {En el nombre del Padre, y del Hijo, y del Esp\'iritu Santo.}

\noindent
\Large{{\bfseries \textcolor{red}{V/.}} \hspace{1cm} Am\'en.}

\noindent
\lettrine[lines=2]{\bfseries \textcolor{red}{H}}{}\Large {ermanos:}\\
\Large {Que Dios abra vuestro coraz\'on a su ley \\
y os conceda la paz; \\
que escuche vuestras oraciones \\
y qued\'eis reconciliados con \'el.}

\noindent
{\bfseries \textcolor{red}{V/.}} \hspace{1cm} \Large Am\'en.

\Large {\bfseries \textcolor{red}{Oraci\'on}}

\large {\textcolor{red}{El sacerdote invita a todos a la oraci\'on, con estas o parecidas palabras:}}

\Large {Oremos, hermanos, para que Dios, que nos llama a la conversi\'on, nos conceda la gracia de una verdadera y fructuosa penitencia.}

\large {\textcolor{red}{Todos oran en silencio durante algunos momentos. Luego, el sacerdote recita la siguiente plegaria:}}

\noindent
\Large {Escucha, Se\~nor, nuestras s\'uplicas humildes\\
y perdona los pecados de quienes nos confesamos culpables\\
para que as\'i podamos recibir tu perd\'on y tu paz.\\
Por Jesucristo nuestro Se\~nor.}

\noindent
\Large {\bfseries \textcolor{red}{V/.}} \hspace{1cm} Am\'en

\begin{center}
\Huge {\bfseries Lit\'urgia de la Palabra}
\end{center}

\large {\textcolor{red}{Comienza ahora la celebraci\'on de la Palabra. Si hay varias lecturas, puede intercalarse entre ellas un salmo, un canto apropiado o un momento de silencio, para conseguir as\'i que la Palabra de Dios sea mejor comprendida por cada uno, y se le preste una mayor adhesi\'on. Si hubiese solamente una lectura, conviene que se tome del Evangelio.}}

\Large {\bfseries \textcolor{red}{Homil\'ia}}

\large {\textcolor{red}{Sigue la homil\'ia que, partiendo del texto de las lecturas, debe conducir a los penitentes al examen de conciencia y a la renovaci\'on de vida.}}

\Large {\bfseries \textcolor{red}{Examen de conciencia}}

\large {\textcolor{red}{Es conveniente que se guarde un tiempo de silencio para examinar la conciencia y suscitar la verdadera contrici\'on de los pecados. El sacerdote o el di\'acono u otro ministro, pueden ayudar a los fieles con breves pensamientos o algunas preces lit\'anicas, teniendo siempre en cuenta su mentalidad, su edad, etc.}}

\newpage

\begin{center}
\Huge {\bfseries Rito de la Reconciliaci\'on}
\end{center}

\Large {\bfseries \textcolor{red}{Confesi\'on general de los pecados}} 

\large {\textcolor{red}{A invitaci\'on del di\'acono o de otro ministro, los asistentes se arrodillan o inclinan, y recitan la confesi\'on general (por ejemplo, Yo pecador). Luego, de pie, si se juzga oportuno se hace alguna oraci\'on lit\'anica o se entona un c\'antico. Al final, se acaba con la oraci\'on dominical, que nunca deber\'a omitirse.}} 

\large {\textcolor{red}{El di\'acono o el ministro:}}

\lettrine[lines=2]{\bfseries \textcolor{red}{H}}{}\Large {ermanos:\\ 
Confesad vuestros pecados y orad unos por otros, para que os salv\'eis.}

\large {\textcolor{red}{Todos juntos dicen:}}\\
\noindent
\Large {Yo confieso ante Dios todopoderoso\\
y ante vosotros, hermanos,\\
que he pecado mucho\\
de pensamiento, palabra, obra y omisi\'on.\\
Por mi culpa, por mi culpa, por mi gran culpa.\\
Por eso ruego a Santa Mar\'ia, siempre Virgen,\\
a los \'Angeles, a los Santos\\
y a vosotros, hermanos,\
que interced\'ais por m\'i ante Dios nuestro Se\~nor.}

\large {\textcolor{red}{El di\'acono o el ministro:}}\\
\noindent
\Large {Pidamos humildemente a Dios misericordioso,\\
que purifica los corazones\\
de quienes se confiesan pecadores\\
y libra de las ataduras del mal\\
a quienes se acusan de sus pecados,\\
que conceda el perd\'on a los culpables\\
y cure sus heridas.}

--- Que nos concedas la gracia de una verdadera penitencia.\\
\noindent
\Large {\bfseries \textcolor{red}{R/.}} \hspace{0.5cm} Te rogamos, \'oyenos.

--- Que nos concedas el perd\'on y borres las deudas de nuestros antiguos pecados.\\
\noindent
\Large {\bfseries \textcolor{red}{R/.}} \hspace{0.5cm} Te rogamos, \'oyenos.

--- Que quienes nos hemos apartado de la santidad de la Iglesia,
consigamos el perd\'on de nuestras culpas y volvamos limpios a ella.\\
\noindent
\Large {\bfseries \textcolor{red}{R/.}} \hspace{0.5cm} Te rogamos, \'oyenos.

--- Que a quienes con el pecado hemos manchado nuestro bautismo,
nos devuelvas a su primitiva blancura.\\
\noindent
\Large {\bfseries \textcolor{red}{R/.}} \hspace{0.5cm} Te rogamos, \'oyenos.

--- Que, al acercarnos de nuevo a tu altar santo, seamos transformados
por la esperanza de la vida eterna.\\
\noindent
\Large {\bfseries \textcolor{red}{R/.}} \hspace{0.5cm} Te rogamos, \'oyenos.

--- Que permanezcamos, de aqu\'i en adelante, con entrega sincera, fieles a tus sacramentos, y mostremos siempre nuestra adhesi\'on a ti.\\
\noindent
\Large {\bfseries \textcolor{red}{R/.}} \hspace{0.5cm} Te rogamos, \'oyenos.

--- Que, renovados en la caridad, seamos testigos de tu amor en el
mundo.\\
\noindent
\Large {\bfseries \textcolor{red}{R/.}} \hspace{0.5cm} Te rogamos, \'oyenos.

--- Que perseveremos fieles a tus mandamientos y lleguemos a la vida
eterna.\\
\noindent
\Large {\bfseries \textcolor{red}{R/.}} \hspace{0.5cm} Te rogamos, \'oyenos.

\newpage

\large {\textcolor{red}{El di\'acono o el ministro:}}\\
\Large {Con las mismas palabras que Cristo nos ense\~n\'o,\\
pidamos a Dios Padre que perdone nuestros pecados\\
y nos libre de todo mal.}

\large {\textcolor{red}{Todos juntos dicen:}}\\
\Large {Padre nuestro, que est\'as en el cielo,\\
santificado sea tu Nombre;\\
venga a nosotros tu reino;\\
h\'agase tu voluntad en la tierra como en el cielo.\\
Danos hoy nuestro pan de cada d\'ia;\\
perdona nuestras ofensas,\\
como tambi\'en nosotros perdonamos\\
a los que nos ofenden;\\
no nos dejes caer en la tentaci\'on,\\
y l\'ibranos del mal.}

\large {\textcolor{red}{El sacerdote concluye, diciendo:}}\\
\Large {Escucha, Se\~nor, a tus siervos,\\
que se reconocen pecadores;\\
y haz que, liberados por tu Iglesia de toda culpa,\\
merezcan darte gracias con un coraz\'on renovado.\\
Por Jesucristo nuestro Se\~nor.}

\large {\textcolor{red}{Todos responden:}}\\
\Large {\bfseries \textcolor{red}{R/.}} \hspace{0.5cm} \Large {Am\'en}

\newpage

\Large {\bfseries \textcolor{red}{Confesi\'on y absoluci\'on individual}}

\large {\textcolor{red}{A continuaci\'on, los fieles se acercan a los sacerdotes que se hallan en lugares adecuados y confiesan sus pecados, de los que son absueltos cada penitente individualmente, una vez impuesta y aceptada la correspondiente satisfacci\'on. Tras la confesi\'on y, si se juzga oportuno, despu\'es de una conveniente exhortaci\'on, omitido todo lo que suele hacerse en la reconciliaci\'on de un solo penitente, el sacerdote, extendiendo ambas manos, o al menos la derecha, sobre la cabeza del penitente, da la absoluci\'on, diciendo:}}

\lettrine[lines=2]{\bfseries \textcolor{red}{D}}{} \Large ios, Padre misericordioso,\\
que reconcili\'o consigo al mundo\\
por la muerte y la resurrecci\'on de su Hijo\\
y derram\'o el Esp\'iritu Santo\\
para la remisi\'on de los pecados,\\
te conceda, por el ministerio de la Iglesia,\\
el perd\'on y la paz.\\
{\bfseries YO TE ABSUELVO DE TUS PECADOS\\
EN EL NOMBRE DEL PADRE, Y DEL HIJO,\\
\Huge{\textcolor{red}{\ding{64}}} \Large Y DEL ESP\'IRITU SANTO.}

\large {\textcolor{red}{El penitente responde:}}\\
\noindent
\Large {\bfseries \textcolor{red}{R/.}} \hspace{0.5cm} \Large {Am\'en}


\Large {\bfseries \textcolor{red}{Acci\'on de gracias por la misericordia de Dios}}

\large {\textcolor{red}{Una vez concluidas las confesiones de los penitentes, el sacerdote que preside la celebraci\'on, teniendo junto a s\'i a los otros sacerdotes, invita a la acci\'on de gracias y a la pr\'actica de las buenas obras, con las que se manifiesta la gracia de la penitencia, tanto en la vida de cada uno como en la de la comunidad. Es conveniente que todos juntos canten alg\'un salmo o himno apropiado, o bien que se haga una oraci\'on lit\'anica, para proclamar el poder y la misericordia de Dios.}}

\newpage

\Large {\bfseries \textcolor{red}{Oraci\'on final de acci\'on de gracias}}

\noindent
\Large {\bfseries \textcolor{red}{V/.}} \hspace{0.5cm} El Se\~nor est\'e con vosotros.\\
\noindent
\Large {\bfseries \textcolor{red}{R/.}} \hspace{0.5cm} Y con tu esp\'iritu.

\noindent
\Large {\bfseries \textcolor{red}{V/.}} \hspace{0.5cm} Levantemos el coraz\'on.\\
\noindent
\Large {\bfseries \textcolor{red}{R/.}} \hspace{0.5cm} Lo tenemos levantado hacia el Se\~nor. 

\noindent
\Large {\bfseries \textcolor{red}{V/.}} \hspace{0.5cm} Demos gracias al Se\~nor, nuestro Dios.\\
\noindent
\Large {\bfseries \textcolor{red}{R/.}} \hspace{0.5cm} Es justo y necesario.

\lettrine[lines=2]{\bfseries \textcolor{red}{R}}{}\Large Realmente es justo y necesario,\\
es nuestro deber y salvaci\'on\\
glorificarte, siempre Se\~nor,\\
que admirablemente has creado al hombre,\\
y m\'as admirablemente has hecho en \'el\\
una nueva creaci\'on.

\noindent
\Large T\'u, no abandonas al pecador,\\
sino que lo llamas por la fuerza de tu amor.\\
T\'u, has enviado a tu Hijo al mundo,\\
para destruir el pecado y la muerte,\\
y en su resurrecci\'on\\
no has devuelto la vida y la alegr\'ia.

\lettrine[lines=2]{\bfseries \textcolor{red}{T}}{}\Large \'u, nos renuevas por la fuerza del Evangelio\\
y de los Sacramentos.

\noindent
\Large T\'u, has derramado el Esp\'iritu Santo\\
en nuestros corazones,\\
para hacernos herederos e hijos tuyos.

\noindent
\Large T\'u, has derramado el Esp\'iritu Santo\\
en nuestros corazones,\\
para hacernos herederos e hijos tuyos.

\noindent
\Large T\'u, nos libras de la esclavitud del pecado\\
y nos transformas d\'ia a d\'ia\\
en la imagen de tu Hijo.

\lettrine[lines=2]{\bfseries \textcolor{red}{A}}{}\Large labamos y bendicimos tu nombre\\
y te damos gracias\\
por las maravillas de tu misericordia.

\noindent
\Large Y con los \'angeles y los santos,\\
cantamos, cantamos\\
el himno de tu gloria.
    
\large{\textcolor{red}{Con las manos juntas y en uni\'on del pueblo cantan o recitan en voz alta:}}

\noindent
\Large {Santo, Santo, Santo es el Se\~nor, Dios del universo.\\
Llenos est\'an el cielo y la tierra de tu gloria.\\
Hosanna en el cielo.\\
Bendito el que viene en nombre del Se\~nor.\\
Hosanna en el cielo.}

\Large {\bfseries \textcolor{red}{Rito de la Paz}}

\large {\textcolor{red}{El celebrante, extendiendo y juntando las manos, dice:}}

\noindent
\Large {\bfseries \textcolor{red}{V/.}} \hspace{0.5cm} La paz del Se\~nor est\'e siempre con ustedes.\\
\noindent
\Large {\bfseries \textcolor{red}{R/.}} \hspace{0.5cm} Y con tu esp\'iritu. 

\noindent
\Large {\bfseries \textcolor{red}{V/.}} \hspace{0.5cm} Dense fraternalmente la paz. 

\begin{center}
\Huge {\bfseries Rito de Concluci\'on}
\end{center}

\large{\textcolor{red}{En este momento se hacen, si es necesario y con brevedad, los oportunos anuncios o advertencias al pueblo.}}

\large{\textcolor{red}{Despu\'es tiene lugar la despedida. El sacerdote bendice a todos, diciendo:}}

\noindent
\Large {\bfseries \textcolor{red}{V/.}} \hspace{0.5cm} El Se\~nor dirija vuestros corazones en la caridad de Dios y en la espera de Cristo.\\
\noindent
\Large {\bfseries \textcolor{red}{R/.}} \hspace{0.5cm} Am\'en.

\large{\textcolor{red}{Celebrante:}}\\
\Large {\bfseries \textcolor{red}{V/.}} \hspace{0.5cm} Para que pod\'ais caminar con una vida nueva y agradar a Dios en todas las cosas.\\
\noindent
\Large {\bfseries \textcolor{red}{R/.}} \hspace{0.5cm} Am\'en.

\lettrine[lines=2]{\bfseries \textcolor{red}{L}}{} \Large a bendici\'on de Dios todopoderoso, \\
Padre, Hijo \Huge{\textcolor{red}{\ding{64}}} \Large y Esp\'iritu Santo, \\
descienda sobre vosotros y os acompa\~ne siempre.

\noindent
\Large {\bfseries \textcolor{red}{R/.}} \hspace{0.5cm} Am\'en.

\large{\textcolor{red}{Luego el celebrante, con las manos juntas, despide al pueblo:}}

\lettrine[lines=2]{\bfseries \textcolor{red}{E}}{}\Large l Se\~nor ha perdonado vuestros pecados.\\
Pod\'eis ir en paz.

\noindent
\Large {\bfseries \textcolor{red}{R/.}} \hspace{0.5cm} Demos gracias a Dios.

\large{\textcolor{red}{Puede utilizarse cualquier otra f\'ormula conveniente.}}

% Termina el documento %%%%%%%%%%%%%%%%%%%%%%%%%%%%% F I N %%%%%%%%%%%%%%%%%%%%%%%%%%%%%%%%%%%%%%%%%%%%%%%%%%%%%%
\end{document}
