% ********************************************************
%   MISA EN HONOR A LA FIESTA DE CRISTO REY DEL UNIVERSO
%   Texto y musica con acompañamiento
%   by serachsam
% ********************************************************

\documentclass[letterpaper]{report}

\usepackage[spanish]{babel}
\usepackage[T1]{fontenc}
\usepackage{pifont}
\usepackage{pdflscape}
% \usepackage{lscape}
\usepackage{geometry}
\geometry{
    letterpaper,
    left=1.5cm,
    right=1.5cm,
    bottom=1.5cm,
    top=1.5cm
}

\setlength{\parskip}{\baselineskip}

%% Portada del Libro
\title{ \textbf{ \Huge \scshape Misa Cristo Rey  } \\
{ \LARGE Solemnidad Cristo Rey del Universo } }
\author{ \textit{ \large Carmelitas Descalzas, Managua, Nicaragua } }
\date{ \LARGE Linda Isabel Martínez Castro \\ Samuel José Gutiérrez Avilés \\ \small \textit{2018 - 2020} }

\begin{document}
    \pagenumbering{gobble}
    %\pagestyle{plain}

    %% - Portada
    \maketitle

    \begin{titlepage}
        \centering
        \vspace*{8cm}
        { \scshape \Huge Ordinario de la Misa \par}
        \vspace{1cm}
        { \itshape \Large XXXIV Domingo del Tiempo Ordinario \par}
        \vfill
    \end{titlepage}
    \newpage

    %% - Kyrie eleison - Melodia a modo del renacimiento
    \begin{center}
        {\scshape \Huge {\bfseries Se\~nor ten piedad}} \\
        {\LARGE {\bfseries Misa Cristo Rey}} \\
        {\Large {\bfseries Kyrie Eleison}}
    \end{center}
    
    \LARGE {T\'u que nos libraste del pecado y de la muerte.}
    
    \LARGE \begin{tabular}{ll}
        Se\~nor ten piedad, piedad.& Se\~nor ten piedad.
    \end{tabular}
    
    \LARGE {T\'u que nos reconciliaste con el Padre y nuestros hermanos.}
    
    \LARGE \begin{tabular}{ll}
        Cristo ten piedad, piedad.& Cristo ten piedad.
    \end{tabular}
    
    \LARGE {T\'u que nos resucitar\'as y glorificar\'as contigo.}
    
    \LARGE \begin{tabular}{ll}
        Se\~nor ten piedad, piedad.& Se\~nor ten piedad.
    \end{tabular}
    
    \clearpage

    %% - Gloria in excelsis Deo - Melodia a modo del renacimiento
    \begin{center}
        {\scshape \Huge {\bfseries Gloria a Dios}} \\
        {\LARGE {\bfseries Misa Cristo Rey}} \\
        {\Large {\bfseries Kyrie Eleison}}
    \end{center}
    
    \LARGE Gloria a Dios en lo alto del cielo.\\
    Y en la tierra paz a los hombres que ama el Se\~nor. \\
    Te alabamos, te glorificamos, te damos gracias por tu gloria.
    
    Se\~nor Dios, Rey celestial, Dios Padre todopoderoso. \\
    Se\~nor, Hijo \'unico, Jesucristo.
    
    Se\~nor Dios, Cordero de Dios, Hijo del Padre, \\
    T\'u que quitas el pecado del mundo, ten piedad de nosotros.
    
    T\'u que quitas el pecado del mundo, atiende a nuestra s\'uplica. \\
    T\'u, que est\'as sentado a la derecha del Padre, ten piedad de nosotros.
    
    Porque s\'olo T\'u eres Santo, Se\~nor alt\'isimo Jesucristo. \\
    Con el Esp\'iritu Santo en la gloria.
    
    Am\'en.
    \clearpage

    %% - Credo in unum Deum - Melodia a modo del renacimiento
    \begin{center}
        {\scshape \Huge {\bfseries Creo en Dios}} \\
        {\LARGE {\bfseries Misa Cristo Rey}} \\
        {\Large {\bfseries Credo}}
    \end{center}
    
    \LARGE Creo en Dios. \\
    Padre todopoderoso creador del cielo y de la tierra.
    
    Creo en Jesucristo. \\
    Hijo \'unico, nuestro Se\~nor, que fue concebido por obra y gracia del Esp\'iritu Santo, nacio de Santa Mar\'ia Virgen.
    
    Padeci\'o bajo el poder de Poncio Pilato, fue crucificado, muerto y sepultado. Descendi\'o a los infiernos, al tercer d\'ia resucit\'o de entre los muertos, resucit\'o de entre los muertos.
    
    Subi\'o al cielo y est\'a sentado a la derecha de Dios Padre todopoderoso. Desde all\'i ha de venir a juzgar a vivos y a muertos.
    
    Creo en el Esp\'iritu Santo. \\
    La santa Iglesia cat\'olica, la comuni\'on de los santos, el perd\'on de los pecados, la resurrecci\'on de la carne y en la vida eterna.
    
    Am\'en.
    \clearpage

    %% - Sanctus - Melodia a modo del renacimiento
    \begin{center}
        {\scshape \Huge {\bfseries Santo}} \\
        {\LARGE {\bfseries Misa Cristo Rey}} \\
        {\Large {\bfseries Sanctus}}
    \end{center}
    
    \LARGE Santo.
    
    Santo, santo, santo. \\
    Los cielos y la tierra est\'an llenos de tu gloria.\\
    Hosana, hosana, en el cielo.

    Santo, santo, santo. \\
    Bendito el que viene en el nombre del Se\~nor. \\
    Hosana, hosana, en el cielo.
    \clearpage

    %% - Agnus Dei - Melodia a modo del renacimiento
    \begin{center}
        {\scshape \Huge {\bfseries Cordero de Dios}} \\
        {\LARGE {\bfseries Misa Cristo Rey}} \\
        {\Large {\bfseries Agnus Dei}}
    \end{center}
    
    \LARGE Cordero de Dios.
    
    \LARGE \begin{tabular}{ll}
        Que quitas el pecado del mundo.& Ten piedad de nosotros.
    \end{tabular}
    
    \LARGE Cordero de Dios.
    
    \LARGE \begin{tabular}{ll}
        Que quitas el pecado del mundo.& Ten piedad de nosotros.
    \end{tabular}
    
    \LARGE Cordero de Dios.
    
    \LARGE \begin{tabular}{ll}
        Que quitas el pecado del mundo.& Danos la paz.
    \end{tabular}
    \clearpage

    \begin{titlepage}
        \centering
        \vspace*{8cm}
        { \scshape \Huge Propio de la Misa \par}
        \vspace{1cm}
        { \itshape \Large XXXIV Domingo del Tiempo Ordinario \par}
        \vfill
    \end{titlepage}
    \clearpage

    %% - Introito - Melodia a inspiracion de Mnsr. Marco Frisina
    \begin{center}
        {\scshape \Huge {\bfseries Entrada}} \\
        {\LARGE {\bfseries Misa Cristo Rey}} \\
        {\Large {\bfseries Introito - Principe de los siglos}}
    \end{center}
    
    \LARGE Pr\'incipe absoluto de los siglos,\\
    Jesucristo, rey de las naciones:\\
    te confesamos \'arbitro supremo\\
    de las mentes y los corazones.
    
    En la tierra te adoran los mortales\\
    y los santos te alaban en el cielo,\\
    unidos a sus voces te aclamamos\\
    proclam\'andote rey del universo.
    
    Jesucristo, pr\'incipe pac\'ifico:\\
    somete a los esp\'iritus rebeldes,\\
    haz que encuentren el rumbo los perdidos\\
    en un solo aprisco se congreguen.
    
    Por eso pendes de una cruz sangrienta,\\
    abres en ella tus divinos brazos;\\
    por eso muestras en tu pecho herido\\
    tu ardiente coraz\'on atravesado.
    
    Est\'as oculto en los altares\\
    tras las im\'agenes del pan y el vino;\\
    por eso viertes de tu pecho abierto\\
    sangre de salvaci\'on para tus hijos.
    
    Con honores p\'ublicos te ensalcen\\
    los que tienen poder sobre la tierra;\\
    El maestro y el juez te rindan  culto,\\
    el arte y la ley no te desmientan.\\
  
    Las insignias de los reyes todos\\
    te sean para siempre dedicadas,\\
    y est\'en sometidos a tu cetro\\
    los ciudadanos de las naciones.
  
    Gobiernas con amor el universo,\\
    glorificado seas, Jesucristo,\\
    y que contigo y con tu eterno Padre\\
    reciba gloria el Santo Esp\'iritu.
  
    Am\'en.
    \clearpage
    
    %% - Offertorium - Melodia a inspiracion de Mnsr. Marco Frisina
    \begin{center}
        {\scshape \Huge {\bfseries Presentaci\'on de ofrendas}} \\
        {\LARGE {\bfseries Misa Cristo Rey}} \\
        {\Large {\bfseries Offertorium - Sagrario del Altar}}
    \end{center}
    
    Sagrario del Altar, Sagrario del Altar, nido de tu tierno amor.
    
    Sagrario del Altar, Sagrario del Altar, nido de tu tierno amor.
    
    Tu amor, es amor de cielo,\\
    mi amor, mezcla de cielo y tierra.\\
    Tu amor, es puro e infinito,\\
    m\'i amor, limitado e imperfecto.
    
    Sagrario del Altar, Sagrario del Altar, nido de tu tierno amor.
    
    Sea yo, Jes\'us m\'io, desde hoy,\\
    todo para Ti, como T\'u para mi.\\
    Que te ame yo siempre,\\
    como te amaron los Ap\'ostoles;
  
    Sagrario del Altar, Sagrario del Altar, nido de tu tierno amor.

    Mis labios besen tus pies,\\
    como los bes\'o la Magdalena.\\
    Mira y escucha mi coraz\'on,\\
    como escuchaste a Zaqueo.

    Sagrario del Altar, Sagrario del Altar, nido de tu tierno amor.

    Amor me pides y amor me das.\\
    D\'ejame reclinarme en tu pecho\\
    como a tu disc\'ipulo amado.\\
    Deseo vivir contigo.

    Sagrario del Altar, Sagrario del Altar, nido de tu tierno amor.\\

    S\'olo tu amor, mi amado,\\
    en Ti mi vida puse.\\
    Para el mundo soy una flor marchita,\\
    no quiero m\'as que am\'andote, morir.

    Sagrario del Altar, Sagrario del Altar, nido de tu tierno amor.
    \clearpage

    %% - Communio - Melodia a inspiracion de Mnsr. Marco Frisina
    \begin{center}
        {\scshape \Huge {\bfseries Comuni\'on}} \\
        {\LARGE {\bfseries Misa Cristo Rey}} \\
        {\Large {\bfseries Communio - Madre de la Iglesia}}
    \end{center}
    
    Cabeza y cuerpo, Cristo forma un todo,\\
    hijo de Dios e hijo de Mar\'ia:\\
    un hijo en quien se juntan muchos hijos:\\
    en su Madre la Iglesia se perfila.

    Una y otra son madre y son v\'irgen,\\
    una y otra del Esp\'iritu conciben,\\
    una y otra sin mancha ni pecado,\\
    al Padre celestial engendran hijos.

    Mar\'ia da al cuerpo la cabeza,\\
    la Iglesia a la cabeza da el cuerpo:\\
    una y otra son madre del Se\~nor,\\
    ninguna sin la otra por entero.

    Gloria a la Trinidad inaccesible\\
    que ha querido morar entre nosotros,\\
    en Mar\'ia, la Iglesia, en nuestra alma,\\
    para llenarnos de su eterno gozo.

    Am\'en.
    \clearpage

    %% - Finalis - Melodia a inspiracion de Mnsr. Marco Frisina
    \begin{center}
        {\scshape \Huge {\bfseries Final}} \\
        {\LARGE {\bfseries Misa Cristo Rey}} \\
        {\Large {\bfseries Finalis - Canto de amor}}
    \end{center}
    
    Eres el m\'as hermoso\\
    de los hijos de Ad\'an,\\
    la gracia est\'a en tus labios.\\
    Dios te bendijo para siempre.
  
    Ci\~ne tu espada al costado,\\
    en tu gloria y tu esplendor\\
    cabalga por la verdad,\\
    la piedad y la justicia.
  
    !`Tensa la cuerda en el arco,\\
    que hace terrible a tu derecha!\\
    Agudas son tus flechas,\\
    bajo tus pies est\'an los pueblos.
  
    Tu trono es de Dios;\\
    tu cetro es la equidad;\\
    t\'u amas la justicia\\
    y odias la impiedad.
  
    Dios te ha ungido con \'oleo\\
    Desde palacios la\'udes te recrean.\\
    Princesas son tus preferidas;\\
    a tu diestra est\'a la reina.
  
    Escucha hija pon o\'ido,\\
    olvida la casa de tu padre,\\
    el rey se prendar\'a de t\'i.\\
    El es tu Se\~nor. !`P\'ostrate ante \'el!\\
  
    La hija de Tiro con presentes,\\
    toda espl\'endida, la hija del rey,\\
    con vestidos en oro;\\
    es llevada ante el rey.
  
    V\'irgenes tras ella,\\
    donde \'el son llevadas;\\
    entre alborozo avanzan,\\
    entran en el palacio del rey.
  
    En lugar de padres tendr\'as hijos;\\
    pr\'incipes los har\'as de la tierra.\\
    !`Tu nombre ser\'a memorable,\\
    los pueblos te alabaran por los siglos!
    \clearpage
    
    \begin{titlepage}
        \centering
        \vspace*{8cm}
        { \scshape \Huge Cantillatio \par}
        \vspace{1cm}
        { \itshape \Large XXXIV Domingo del Tiempo Ordinario \par}
        \vfill
    \end{titlepage}
    \clearpage
    
    %% - Graduale - Melodia a inspiracion del Rvd. Lucien Deiss
    \begin{center}
        {\scshape \Huge {\bfseries Salmo responsorial}} \\
        {\LARGE {\bfseries Misa Cristo Rey}} \\
        {\Large {\bfseries Melodia responsorial - Ciclo A}}
    \end{center}
    
    El Se\~nor es mi pastor, nada me falta.
    
    El Se\~nor es mi pastor, nada me falta.
    
    El Se\~nor es mi pastor, nada me falta:\\
    en verdes praderas me hace recostar.
    
    El Se\~nor es mi pastor, nada me falta.
    
    Me conduce hacia fuentes tranquilas\\
    y repara mis fuerzas;\\
    me gu\'ia por el sendero justo,\\
    por el honor de su nombre.
    
    El Se\~nor es mi pastor, nada me falta.
    
    Preparas una mesa ante m\i,\\
    enfrente de mis enemigos;\\
    me unges la cabeza con perfume,\\
    y mi copa rebosa.

    El Se\~nor es mi pastor, nada me falta.

    Tu bondad y tu misericordia me acompa\~nan,\\
    todos los d\'ias de mi vida,\\
    y habitar\'e en la casa del Se\~nor,\\
    por a\~nos sin t\'ermino.

    El Se\~nor es mi pastor, nada me falta.
    \clearpage
    
    \begin{center}
        {\scshape \Huge {\bfseries Salmo responsorial}} \\
        {\LARGE {\bfseries Misa Cristo Rey}} \\
        {\Large {\bfseries Melodia responsorial - Ciclo B}}
    \end{center}
    
    El Se\~nor reina, vestido de majestad.
    
    El Se\~nor reina, vestido de majestad.

    El Se\~nor reina, vestido de majestad,\\
    el Se\~nor, vestido y ce\~nido de poder.

    El Se\~nor reina, vestido de majestad.

    As\'i est\'a firme el orbe y no vacila.\\
    Tu trono est\'a firme desde siempre,\\
    y t\'u eres eterno.

    El Se\~nor reina, vestido de majestad.

    Tus mandatos son fieles y seguros;\\
    la santidad es el adorno de tu casa,\\
    Se\~nor, por d\'ias sin t\'ermino.

    El Se\~nor reina, vestido de majestad.
    \clearpage
    
    \begin{center}
        {\scshape \Huge {\bfseries Salmo responsorial}} \\
        {\LARGE {\bfseries Misa Cristo Rey}} \\
        {\Large {\bfseries Melodia responsorial - Ciclo C}}
    \end{center}
    
    Vamos alegres a la casa del Se\~nor.
    
    Vamos alegres a la casa del Se\~nor.

    !`Qu\'e alegr\'ia cuando me dijeron:\\
    <<Vamos a la casa del Se\~nor>>!\\
    Ya est\'an pisando nuestros pies\\
    por el honor de su nombre.

    Vamos alegres a la casa del Se\~nor.

    All\'a suben las tribus,\\
    las tribus del Se\~nor,\\
    seg\'un la costumbre de Israel,\\
    a celebrar el nombre del Se\~nor;\\
    en ella est\'an los tribunales de justicia,\\
    en el palacio de David.

    Vamos alegres a la casa del Se\~nor.
    \clearpage
    
    %% - Aleluya - Melodia a inspiracion del Rvd. Lucien Deiss
    \begin{center}
        {\scshape \Huge {\bfseries Aleluya}} \\
        {\LARGE {\bfseries Misa Cristo Rey}} \\
        {\Large {\bfseries Aclamaci \'on al Evangelio}}
    \end{center}
    
    Aleluya, aleluya, aleluya.
  
    Bendito el que viene en nombre del Se\~nor.\\
    Bendito el reino que llega, el de nuestro padre David.
  
    Aleluya, aleluya, aleluya.
    \clearpage


\end{document}
